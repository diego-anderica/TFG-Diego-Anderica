\chapter{Resultados}
\label{chap:resultados}

\drop{D}{urante} el desarrollo de este capítulo se mostrarán los resultados obtenidos de seguir la planificación usando la metodología de desarrollo detallada en el capítulo anterior, por lo que el avance se dividirá en Sprints. Cada uno de ellos tendrá, al menos, una historia de usuario asignada que deberá quedar implementada al final del desarrollo.

Debido a que el equipo de Scrum, en este caso, es reducido, se han tenido que realizar diversos ajustes para adoptar esta metodología al trabajo que se va a desarrollar. Por tanto, el equipo identificado queda de la siguiente manera:

\begin{itemize}
	\item \textbf{Dueño del Producto:} Luis Rodríguez Benítez.
	\item \textbf{Maestro de Scrum:} Luis Jiménez Linares.
	\item \textbf{Equipo de Scrum:} Diego Andérica Richard.
\end{itemize}

Este capítulo constará de los diferentes Sprints en los que se ha dividido el proyecto, así como su planificación, resultados y reajustes necesarios, si los hubiese. Antes de comenzar con el primer Sprint se ha de tener en cuenta uno de los principales artefactos de la metodología escogida, como es la Pila de Producto.

\section{Planificación inicial}
\subsection{Pila de Producto}
La Pila de Producto, como se ha explicado con anterioridad, se trata de una lista de todo lo que puede ser necesario en el producto. En este caso, la Pila de Producto es la siguiente:

\section{Sprint 1: Diseño de una plataforma Web para la gestión de los usuarios}
Este primer Sprint está enfocado a diseñar e implementar una plataforma Web para la gestión de los usuarios de la aplicación Android. De esta manera, se busca que el hecho de realizar tareas como las de añadir nuevos usuarios, consultar los existentes o modificarlos resulte mucho más sencillo, cómodo y práctico para los potenciales administradores de la plataforma de mensajería instantánea que se pretende desarrollar, puesto que ofrecerá una interfaz más amigable que la que ofrece actualmente la base de datos de segunda generación de Firebase, Firestore.

\subsection{Planificación del Sprint}
Se ha elegido una única historia de usuario para la consecución de este Sprint durante la reunión inicial, representada en la Tabla \ref{tab:historia1}.

\begin{table}[hp]
	\centering
	{\small
		\resizebox{15cm}{!} {
	\begin{tabular}{|l|l|}
		\hline
		\multicolumn{2}{|c|}{\cellcolor[HTML]{343434}{\color[HTML]{FFFFFF} \textbf{Historia de Usuario}}} \\
		\hline
		\multicolumn{2}{|c|}{\textbf{Sprint Asignado:} 1.} \\
		\hline
		\textbf{Número de Historia:} 1. & \textbf{Usuario/Rol:} Administrador.\\
		\hline
		\multicolumn{2}{|l|}{\textbf{Nombre de la Historia:} Gestión de usuarios de la aplicación móvil.} \\
		\hline
		\textbf{Prioridad:} Alta & \textbf{Duración:} 30 horas.\\
		\hline
		\multicolumn{2}{|l|}{\textbf{Descripción:} Desarrollar una plataforma Web para facilitar la gestión de las familias.} \\
		\hline
		\specialcell{\underline{\textbf{Tareas}} \\ Diseño inicial de la base de datos. \\ Implementación de la plataforma Web. \\ Integrar Bootstrap y diseñar apariencia Web. \\ Conectar aplicación Web con Firebase.} & \specialcell{\underline{\textbf{Pruebas}} \\ Comprobar inserción de datos no válidos. \\ Comprobar alta de usuarios desde \acs{CSV}. \\ Intentar acceder sin identificar a la Web. \\ Comprobar que la Web es \textit{responsive}.} \\
		\hline
	\end{tabular}
}
	}
	\caption[Historia de Usuario 1]
	{Historia de Usuario 1}
	\label{tab:historia1}
\end{table}

%TODO: Extracto código algoritmo MD5
%TODO: LOPD
\subsection{Resultados del Sprint}
\subsubsection{Tarea de diseño de la base de datos}
Firebase Firestore ofrece una base de datos no relacional, lo que significa que no existen relaciones ni tablas como sí existen en las relacionales. Por tanto, en este caso, se tienen colecciones de documentos que, a su vez, pueden tener colecciones de documentos anidadas. En consecuencia, se ha realizado un primer diseño en el que la base de datos tendrá tres colecciones: <<Usuarios>>, para los usuarios de las familias; <<UsuariosWeb>>, que albergará los administradores del sitio Web y <<Docentes>>, que contendrá los datos de los docentes y potenciales administradores de los grupos de chat.

En cuanto a contraseñas, no es necesario guardarlas en la base de datos, a excepción de las de los administradores puesto que, al hacer uso de la funcionalidad de Firebase \textit{Authentication}, éstas se guardan internamente en el proyecto y únicamente son conocidas por la herramienta, asignando además un identificador único de usuario a cada uno.

\subsubsection{Tareas de implementación de la plataforma Web}
En el centro habrá uno o varios administradores de la aplicación que se encargarán de gestionar los usuarios de la aplicación. Por tanto, se contemplan las siguientes acciones: dar de alta, dar de baja, consultar y modificar. La plataforma Web debe proporcionar estas acciones de una manera sencilla, vistosa y \textit{responsive}. Para este propósito se ha decidido usar Bootstrap \cite{Bootstrap}, que proporciona un conjunto de herramientas para desarrollo con HTML, CSS y JavaScript. Del mismo modo, se ha requerido incluir Firebase y las referencias a la base de datos de Firestore del proyecto. Por seguridad, el registro de un nuevo administrador en la base de datos se deberá llevar a cabo de manera manual, accediendo al proyecto de Firebase, donde se guardará un nuevo documento con el correo electrónico y la contraseña cifrada mediante el algoritmo MD5. Del mismo modo, se ha implementado un método para evitar que un usuario acceda a una página conocida dentro del servidor mediante el uso de \textit{Session Storage}. De esta manera, cuando el usuario accede correctamente con su correo y contraseña, el servidor genera y devuelve un número de sesión que se almacena en esta caché del navegador junto con el correo por lo que, si alguno de estos campos se encuentra sin definir en el momento de acceder a una página, se devuelve automáticamente al inicio de sesión. Una vez se ha accedido, el administrador accede a una página principal con cuatro botones y una barra de navegación superior desde donde puede realizar cuatro acciones principales: dar de alta, dar de baja, consultar usuarios y modificar usuarios. En los siguientes apartados se llevará a cabo una explicación de cada uno.

\noindent \underline{Dar de Alta} \newline
En esta primera funcionalidad, el administrador tiene la opción de rellenar un sencillo formulario para introducir un nuevo usuario en la base de datos que se puede rellenar con los datos de un único tutor legal del alumno o, si lo hubiese, también con los del segundo tutor. Los datos que se han especificado para cada tutor son: nombre, primer apellido, segundo apellido, número de teléfono y correo electrónico. Por último, para identificar a este usuario en la base de datos, se le asignará un identificador compuesto de los apellidos de ambos tutores o, en su defecto, del único tutor, añadiendo un número al final. Es decir, si dos familias diferentes poseen los mismos apellidos, se añadirá un número al final de la composición de sus apellidos para el identificador, por lo que quedarán representadas de manera única en la base de datos.

Por otra parte, además de tener el formulario para introducir nuevos usuarios, el administrador también tiene la posibilidad de importar un archivo \acs{CSV} para realizar esta tarea de una manera más sencilla y automatizada. Este archivo debe tener un formato específico detallado en la primera línea del mismo y que consta, principalmente, de los mismos datos que en el formulario pero separados por comas y sin espacios entre ellos, separando los usuarios por un salto de línea.

\noindent \underline{Dar de Baja} \newline
Al entrar en esta página, se dispone de una lista desplegable en la que se cargarán todos los usuarios registrados en la base de datos Web el diseño se basa en tener una lista desplegable con todos los usuarios de la base de datos. Al seleccionar uno de ellos, automáticamente se cargarán en una tabla los datos del mismo para poder ver el usuario que se desea eliminar. Como medida adicional contra el borrado accidental, se preguntará si realmente se desea borrar el usuario seleccionado.

\noindent \underline{Consultar Usuarios} \newline
De manera similar al apartado anterior, se ha implementado una tabla para consultar los datos de los usuarios aunque, en este caso, se dispone de una herramienta básica de búsqueda en la que se puede filtrar por cada uno de los campos existentes en la base de datos. Si se pulsa el botón de buscar sin escribir nada en el filtro, la tabla se cargará con todos los registros existentes.

\noindent \underline{Modificar Usuarios} \newline
Esta funcionalidad se ha implementado mediante el uso de una lista desplegable que carga todos los usuarios de la base de datos y un formulario que inicialmente se muestra bloqueado para no escribir de manera no intencionada y modificar los datos. Al seleccionar uno de estos usuarios, su información se carga en el formulario para realizar una primera inspección de la misma y, al pulsar el botón de modificar, se habilitan los campos para la escritura. Cuando el administrador finalice la modificación de la información, pulsará de nuevo un botón, siendo preguntado sobre si quiere realmente llevar a cabo la modificación. En caso positivo, se sobreescribirá la información el el usuario seleccionado de la base de datos.

\section{Ejemplo Historia de Usuario}

\begin{table}[hp]
	\centering
	{\small
		\resizebox{15cm}{!} {
	\begin{tabular}{|l|l|}
		\hline
		\multicolumn{2}{|c|}{\cellcolor[HTML]{343434}{\color[HTML]{FFFFFF} \textbf{Historia de Usuario}}} \\
		\hline
		\multicolumn{2}{|c|}{\textbf{Sprint Asignado:} 1.} \\
		\hline
		\textbf{Número de Historia:} 1. & \textbf{Usuario/Rol:} Administrador.\\
		\hline
		\multicolumn{2}{|l|}{\textbf{Nombre de la Historia:} Gestión de usuarios de la aplicación móvil.} \\
		\hline
		\textbf{Prioridad:} Alta & \textbf{Duración:} 30 horas.\\
		\hline
		\multicolumn{2}{|l|}{\textbf{Descripción:} Desarrollar una plataforma Web para facilitar la gestión de las familias.} \\
		\hline
		\specialcell{\underline{\textbf{Tareas}} \\ Diseño inicial de la base de datos. \\ Implementación de la plataforma Web. \\ Integrar Bootstrap y diseñar apariencia Web. \\ Conectar aplicación Web con Firebase.} & \specialcell{\underline{\textbf{Pruebas}} \\ Comprobar inserción de datos no válidos. \\ Comprobar alta de usuarios desde \acs{CSV}. \\ Intentar acceder sin identificar a la Web. \\ Comprobar que la Web es \textit{responsive}.} \\
		\hline
	\end{tabular}
}
	}
	\caption[Historia de Usuario 2]
	{Historia de Usuario 2}
	\label{tab:historia2}
\end{table}