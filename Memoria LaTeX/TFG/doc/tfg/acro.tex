\chapter{Listado de acrónimos}

{\small
\begin{acronym}[XXXXXXXX]
  \Acro{GNU}     {\acs{GNU} is Not Unix}
  \acro{OO}      {Orientación a Objetos}
  \acro{RPC}     {Remote Procedure Call}
  \acro{MB}      {MegaByte}
  \acro{API}     {Application Programming Interface}
  \Acro{IP}      {Internet Protocol}
  \Acro{VoIP}    {Voice over \acs{IP}}
  \acro{GB}      {GigaByte}
  \acro{P2P}     {Peer to Peer}
  \acro{SMS}     {Short Message Service}
  \acro{NSA}     {National Security Agency}
  \acro{JCCM}    {Junta de Comunidades de Castilla-La Mancha}
  %\acro{IDE}     {Integrated Development Environment}
  \acro{GHz}	 {Gigahercios}
  \acro{CSV}     {Comma Separated Values}
\end{acronym}
}


% \ac{OO}   la primera vez \acf, después \acs
% \acs{OO}  short: OO
% \acf{OO}  full : Object Oriented (OO)
% \acl{OO}  large: Object Oriented
% \acx{OO}         OO (Object Oriented)

% usa \Acro cuando no debe aparecer nunca expandido en el texto

% Local variables:
%   TeX-master: "main.tex"
% End:
