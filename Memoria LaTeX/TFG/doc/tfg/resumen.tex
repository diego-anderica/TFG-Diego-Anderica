\chapter{Resumen}
La comunicación entre las personas es algo indispensable. Desde tiempos muy antiguos el ser humano ha tratado de comunicarse  utilizando diferentes técnicas como la de escribir en la roca del interior de las cuevas, usando señales de humo, enviando palomas mensajeras o, más tarde, escribiendo cartas. Con la evolución de las nuevas tecnologías se introdujeron nuevas formas de comunicarse con otras personas como el correo electrónico, las redes sociales o el chat, siendo muy recurrido y de gran ayuda cuando se necesita o se quiere hablar con otra persona en tiempo real.

El presente \acf{TFG} aborda la problemática del uso inadecuado del chat mediante herramientas de mensajería instantánea cuando su uso se realiza en entornos tan delicados como lo es el educativo, desarrollando una aplicación móvil que cubra unas necesidades acotadas y determinadas, añadiendo funcionalidades para monitorizar y controlar las conversaciones con el fin de supervisar que dicha herramienta se utilice de una manera adecuada y con el fin para el que fue originalmente ideada.


\chapter{Abstract}

English version of the previous page.
