\chapter{Resumen}
El desarrollo del ser humano en sociedad no puede ser entendido sin la comunicación. Desde tiempos muy antiguos, este ha tratado de comunicarse  utilizando diferentes técnicas como la de escribir en la roca del interior de las cuevas, usando señales de humo, enviando palomas mensajeras o, más tarde, escribiendo cartas. Con la evolución de las nuevas tecnologías se introdujeron nuevas formas de comunicarse con otras personas como el correo electrónico, las redes sociales o el chat, siendo muy recurrido y de gran ayuda cuando se necesita o se quiere hablar con otra persona en tiempo real.

En el presente \acf{TFG} se desarrolla una herramienta de mensajería instantánea enfocada al uso en entornos tan delicados como lo es el educativo o escolar, permitiendo la comunicación entre tutores legales y docentes. Además, posee ciertas características que la diferencian del resto de alternativas existentes, como es el uso de los servicios Watson de \acf{IA} pertenecientes a la empresa IBM, que permiten el análisis del tono lingüístico de un texto, lo que permite distinguir el tono con el que se ha enviado un mensaje y categorizarlo, informando a los docentes acerca del mismo de una manera sencilla dentro de un chat.


\chapter{Abstract}

English version of the previous page.
