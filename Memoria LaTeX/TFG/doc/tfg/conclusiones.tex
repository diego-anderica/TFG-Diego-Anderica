\chapter{Conclusiones y Trabajos Futuros}
\label{chap:conclusiones}

\drop{F}{inalmente}, se expondrán las conclusiones extraídas como desarrollo de este \acs{TFG}, así como ciertos trabajos futuros sobre el mismo que podrían realizarse para mejorar o extender algunas de sus funcionalidades o características.

\section{Conclusiones}
Como se ha detallado en el capítulo 2, el principal objetivo era el de implementar una herramienta de mensajería instantánea para el contexto educativo que permitiese la comunicación entre el profesorado y los padres o tutores legales de los alumnos. Se puede concluir que este objetivo ha sido satisfecho, puesto que la aplicación desarrollada cumple con esa función ya que, desde un primer momento, se comenzó a implementar con la premisa de satisfacer dicho objetivo general. Se ha desarrollado un marco de gestión de usuarios vinculado al contexto educativo y se dispone de un entorno de ejecución multiplataforma, ya que se dispone de una Web destinada a la gestión de los diferentes usuarios ligados a los núcleos familiares que serán los que, posteriormente, hagan uso de la aplicación móvil. Se ha tenido muy en cuenta la usabilidad de esta, enfocándose en la sencillez y facilidad de utilización, de manera que resulte una tarea lo menos tediosa posible para la persona encargada de dicha gestión. Igualmente, gracias a los servicios que ofrece IBM Watson a través de la plataforma Bluemix, se puede controlar y visualizar el tono de cada mensaje que se envía en los chats, mostrándose al administrador mediante un código de colores cuando se trata de un mensaje y mediante un porcentaje cuando se trata del cómputo total en ese chat. Asimismo, para facilitar la comunicación de eventos importantes a un grupo, se ha añadido la integración con Google Calendar mediante un botón que será visible únicamente por el administrador del chat, que lanzará esta aplicación preparada con los correos de cada uno de los participantes. Por último, de la misma manera que un docente puede crear una sala de chat con varias familias, es posible la creación de una sala para la comunicación de un único núcleo familiar.

\clearpage

\section{Trabajos Futuros}
De igual manera, durante el desarrollo del trabajo se han ido presentando algunas mejoras que podrían realizarse en un futuro. Una de estas mejoras es la posibilidad de obtener notificaciones \textit{push}. Cuando se tuvo desarrollado el envío y recepción de los mensajes, se alertó de la posible funcionalidad futura de recibir notificaciones cada vez que se recibía un mensaje en un chat, aunque esta función requiere de un servidor dedicado ex profeso a esa función que se encuentre <<escuchando>> de manera permanente y desarrollarlo iba a consumir recursos adicionales, por lo que se decidió anotarlo como un trabajo a realizar en el futuro.