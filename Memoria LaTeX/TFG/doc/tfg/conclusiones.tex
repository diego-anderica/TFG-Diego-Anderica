\chapter{Conclusiones y Trabajos Futuros}
\label{chap:conclusiones}

\drop{F}{inalmente}, se expondrán las conclusiones obtenidas durante el desarrollo de este \acs{TFG}, así como ciertos trabajos futuros sobre el mismo que podrían realizarse para mejorar o extender algunas de sus funcionalidades o características.

\section{Conclusiones}
Como se ha detallado en el Capítulo \ref{chap:objetivos}, el principal objetivo marcado era el de implementar una herramienta de mensajería instantánea para el contexto educativo que permitiese la comunicación entre el profesorado y los padres o tutores legales de los alumnos. Se puede concluir que este objetivo ha sido satisfecho puesto que la aplicación desarrollada cumple con esa función ya que, desde un primer momento, se comenzó a implementar con la premisa de satisfacer dicho objetivo general. Se ha desarrollado un marco de gestión de usuarios vinculado al contexto educativo y se dispone de un entorno de ejecución multiplataforma, ya que se dispone de un sitio web destinado a la gestión de los diferentes usuarios ligados a los núcleos familiares que serán los que, posteriormente, hagan uso de la aplicación móvil. Se ha tenido muy en cuenta la usabilidad de esta, enfocada hacia la sencillez y facilidad de utilización, de manera que resulte una tarea lo menos tediosa posible para la persona encargada de dicha gestión. Igualmente, gracias a los servicios que ofrece IBM Watson a través de la plataforma Bluemix, se realiza una categorización de mensajes y usuarios. De esta manera, se puede controlar y visualizar el tono de cada mensaje que se envía en los chats, mostrándose al administrador mediante un código de colores cuando se trata de un mensaje y mediante un porcentaje cuando se trata del cómputo total en ese chat. Asimismo, para facilitar la comunicación de eventos importantes a un grupo, se ha añadido la integración con Google Calendar mediante un botón que será visible únicamente para el administrador del chat, que lanzará esta aplicación preparada para la creación del evento con los correos de cada uno de los participantes del chat. Por último, de la misma manera que un docente puede crear una sala de chat con varias familias, es posible la creación de una sala privada para la comunicación de un único núcleo familiar.

\clearpage

\section{Trabajos Futuros}
De igual manera, durante el desarrollo del trabajo se han ido presentando algunas mejoras que podrían realizarse en un futuro. Una de estas mejoras es la posibilidad de obtener notificaciones \textit{push} con los nuevos mensajes que se envíen a un chat. Existe la posibilidad de que algunas de estas notificaciones pudieran clasificarse como notificaciones \textit{broadcast}, es decir, para que las reciban todos los usuarios de la aplicación. Esto sería útil, por ejemplo, si se van a llevar a cabo funciones de mantenimiento en la base de datos o se quiere lanzar un comunicado importante a todas las familias.

Otro trabajo futuro sería el de trasladar a un servidor todas las operaciones que actualmente realiza la aplicación Android. Es decir, las tareas de comprobar mensajes, mostrarlos o calcular el color de los mismos serían llevadas a cabo por un servidor externo o \textit{cloud}.

Finalmente, cabe la posibilidad de investigar acerca de las \acs{API} de Amazon Alexa o Google Home con el fin de integrarlas con la aplicación. De esta manera, se podrían dictar nuevos mensajes a través de la voz y, a su vez, los mensajes que se reciban podrán ser reproducidos en algunos de los dispositivos compatibles con esta tecnología, como los altavoces inteligentes de Amazon o Google.


\setcounter{chapter}{5}
\chapter{Conclusions and Future Work}
\drop{F}{inally}, conclusions obtained during the development of this \acs{TFG} will be explained along with some future work to be done in order to improve or extend some of its \mbox{characteristics}.

\section{Conclusions}
As detailed in Chapter \ref{chap:objetivos}, the main objective was to implement an instant messaging tool for the educational context that allows the communication between teachers and fathers or \mbox{legal} guardians of the students. It can be concluded that this objective has been satisfied \mbox{because} the developed application meets this function since it was developed to please that general objective. A general user managing frame linked to the educational context has been developed and a multiplatform environment is available by using a website to manage the family users that will be using the mobile application. It has been taken into account the usability, focusing on the simplicity and being easy to use so the task of management will be the least teadious possible to the person in charged. Also, thanks to the IBM Watson services accessible through the Bluemix platform, users and messages can be categorized. In that way, the tone of each message sent to a chat can be visualized showing it to the chat administrator with the help of a color code. In a similar way, the tone of each user in a chat can be controlled by means of a percentage. Likewise, to ease the important events to a group, Google Calendar integration has been added through a button visible only to the chat administrator. This button will launch the calendar application ready to the event creation with the emails of each chat participant. Lastly, it is possible to create a private chat room to communicate with a single family in the same way than creating a room with several families.

\clearpage

\section{Future Work}
In the same manner, during the development of this work some improvements have been noticed to be done in a future. One of this improvements is the implementation of \textit{push} \mbox{notifications} with new messages sent to a chat. It exists the possibility that some of the notifications could be classified as \textit{broadcast} notifications i.e. all the application users would receive the notification. That would be useful for example if some database maintenance task will be done or a notice is wanted to be delivered to all the families.

Another future work would be to move all the operations done by the mobile application to a server i.e the tasks of check messages, showing them or calculate its color would be done by a external or \textit{cloud} server.

Finally, there is the possibility of researching about the Amazon Alexa or Google Home \acs{API} with the purpose of integrating them in the application. In that way, new messages could be dictated and received ones could be casted by using some of the compatible devices, like the Amazon or Google intelligent speakers.



