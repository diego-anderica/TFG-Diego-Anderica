\chapter{Protección de Datos de Carácter Personal}
\label{chap:lopd}
Como en cualquier sistema informático en el que se manejen datos de carácter personal, se deben aplicar unas normas y conocer el ámbito de aplicación de estas.  En cuanto a la protección de datos, existe un contexto legislativo:

\begin{itemize}
	\item \textbf{Directiva 95/46/CE} del Parlamento Europeo y del Consejo de 24 de octubre de 1995, relativa a la protección de personas físicas en lo que respecta al tratamiento de datos personales y a la libre circulación.
	\item \textbf{Reglamento 2016/679} del Parlamento Europeo y del Consejo de 27 de abril de 2016, relativo a la protección de las personas físicas en lo que respecta al tratamiento de datos personales y a la libre circulación de estos datos y por el que se deroga la Directiva 95/46/CE.
	\item \textbf{Ley Orgánica 15/1999}, de 13 de diciembre, de Protección de Datos de Carácter Personal \acf{LOPD}.
	\item \textbf{Real Decreto 1720/2007}, de 21 de diciembre, por el que se aprueba el Reglamento de desarrollo de la \acs{LOPD}.
\end{itemize}

Conocido el contexto y que un dato personal se define como cualquier información numérica, alfabética, gráfica, fotográfica, acústica o de cualquier otro tipo concerniente a personas físicas identificadas o identificables, se debe tener muy en cuenta la protección de los datos que se manejan en el sistema. Estos datos son los de las familias, así como de los docentes y, en menor medida, de los administradores de la página Web.

Por otra parte, Firebase posee una página Web en la que se detalla la información acerca de la privacidad y seguridad de los datos \cite{Firebase2018}. Desde un primer momento, se comunica que los Google está comprometido con el \acs{RGPD}, siendo el <<procesador de datos>>, mientras que los clientes de esta plataforma son considerados los <<controladores de datos>>. Esto significa que los datos se encuentran bajo el control del cliente, responsable de obligaciones como cumplir con los derechos de una persona en relación con sus datos personales, por lo que el centro educativo responsable deberá ofrecer un nivel adecuado de protección de medidas de seguridad, además de los derechos ARCO (Acceso, Rectificación, Cancelación y Oposición).