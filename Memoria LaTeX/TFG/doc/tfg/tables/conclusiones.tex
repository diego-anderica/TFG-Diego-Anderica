\resizebox{15cm}{!} {
	\begin{tabular}{|l|l|l|}
		\hline
		\multicolumn{3}{|c|}{\cellcolor[HTML]{333333}{\color[HTML]{FFFFFF} \textbf{Desarrollo de los Objetivos}}}                             \\ \hline
		\textbf{Nº}        & \textbf{Nombre}       & \textbf{Observaciones} \\ \hline
		1           & \specialcell{Implementar un marco de gestión \\ de usuarios vinculado al contexto educativo.} & \specialcell{Este objetivo ha sido satisfecho, puesto que se dispone \\ de una plataforma de fácil utilización desde la que gestionar las familias.}                        \\ \hline
		2           & \specialcell{Proporcionar un entorno de \\ ejecución multiplataforma.} & \specialcell{Estrechamente ligado al anterior, se ha desarrollado \\ una plataforma Web desde la que se pueden gestionar \\ los usuarios que, posteriormente, serán los que utilicen la aplicación móvil.}                      \\ \hline
		3           & \specialcell{Implementar un mecanismo de monitorización \\ activa del tipo y contenido de los mensaje}          & Gracias a los servicios que ofrece IBM Watson a través de la \\ plataforma Bluemix, se puede controlar el tono de cada mensaje que se envía en los chats, mostrándose \\ mediante un código de colores al administrador.                       \\ \hline
		4           & Integración de la aplicación con Google Calendar.                                              & Si un administrador desea crear un evento en Google Calendar, tendrá a su disposición un botón con el que abrir directamente la aplicación, lista con los correos electrónicos del chat en el que se encontraba.                       \\ \hline
		5           & Implementar mecanismos que permitan comunicaciones privadas docente-tutores legales del alumno & Un docente podrá crear una sala de chat privada para comunicarse exclusivamente un único núcleo familiar de la misma manera en la que se crean los chats de grupo.                       \\ \hline
	\end{tabular}
}