\begin{tabular}{p{.3\textwidth}p{.3\textwidth}p{.3\textwidth}}
	\tabheadformat
	                     &
	\tabhead{WhatsApp}   &
	\tabhead{Telegram}   \\
	\hline
	\textbf{<<Chats>> grupales} & Hasta 256 personas & Hasta 10.000 personas \\
	\hline
	\textbf{Llamadas integradas} & Sí & Sí \\
	\hline
	\textbf{Videollamadas integradas} & Sí & No \\
	\hline
	\textbf{Posee aplicación de escritorio} & Sí (requiere de teléfono móvil) & Sí (conversaciones en la nube) \\
	\hline
	\textbf{Cifrado} & Signal (de extremo a extremo por defecto en todas las conversaciones) & De extremo a extremo en chats secretos, servidor-cliente en el resto de conversaciones \\
	\hline
	\textbf{Envío de archivos multimedia} & Sí & Sí \\
	\hline
	\textbf{Envío de documentos y otros archivos} & Sí, con límite de 100 \acs{MB} & Sí, hasta 1,5 \acs{GB} y sin extensión específica. \\
	\hline
	\textbf{Mensajes de voz} & Sí & Sí \\
	\hline
	\textbf{Canales} & No & Sí \\
	\hline
	\textbf{Respaldo de <<chats>> en la nube} & Sí (en la versión de escritorio dependen del móvil) & Sí (en los servidores de Telegram) \\
	\hline
	\textbf{<<Chats>> secretos} & No & Sí \\
	\hline
	\textbf{Creación de \textit{bots}} & No & Sí \\
	\hline
	\textbf{Open Source} & No & Sí \\
	\hline
	\textbf{Existencia de una API} & No & Sí \\
	\hline
	\textbf{Multiplataforma} & Sí & Sí \\
	\hline
\end{tabular}

% Local variables:
%   coding: utf-8
%   ispell-local-dictionary: "castellano8"
%   TeX-master: "main.tex"
% End: