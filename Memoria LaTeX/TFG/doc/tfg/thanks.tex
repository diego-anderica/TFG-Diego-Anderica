\chapter{Agradecimientos}

Este trabajo es fruto del camino que emprendí hace ahora prácticamente cuatro años. Por aquel entonces todo esto me parecía muy lejano, quizá demasiado. Comencé este camino con gente a la que ya conocía, aunque también me he encontrado con otras personas que he ido conociendo y algunas a las que había perdido la pista desde hacía años. Con ellas he compartido grandes momentos y muchas horas de clase, y ha sido todo esto lo que las ha convertido no solo en meros compañeros de clase, sino en mis amigos. A estas personas, gracias por haber estado ahí y por seguir haciéndolo.

Durante este camino también he conocido a una gran cantidad de profesores que, de no ser por ellos, probablemente no estaría escribiendo estas palabras. Sobre todo, me gustaría agradecer a Luis Rodríguez y Luis Jiménez por haberme guiado en este proyecto, ayudándome y dándome algunas pistas que seguir.

En cuanto a la familia, me gustaría agradecer, sobre todo, a mi madre, aunque también a mis tíos y a mi familia más cercana por haber estado ahí. De igual manera, me hubiera gustado compartir todo esto con gente que, desafortunadamente, ya no está. Especialmente, con mi padre, que me introdujo en el mundo de los ordenadores desguazándolos, actualizando sus componentes, <<cacharreando>> con programas, sistemas operativos y cosas por el estilo. Aún recuerdo aquella noche en la que nos quedamos de madrugada intentando instalar \mbox{Windows} XP con disquetes de 3,5'' o cuando creí estropear el portátil nuevo cuando únicamente se había bloqueado. Siendo profesor de informática, tecnología y coordinador TIC's del Colegio <<San José>> de Ciudad Real, sería raro que yo no hubiera escogido una carrera similar a la de ingeniería informática. Estoy muy orgulloso de todo lo que me pudo enseñar durante mis 16 primeros años, del camino en el que pareció introducirme desde pequeño y de él en general. Una vez, alguien me dijo que <<si alguien se ha ido, quizá era porque ya había hecho todo lo que tenía que hacer>>. Ojalá llegue a hacer algún día la mitad de todo lo que hizo él. \linebreak Gracias, papá. 

\quoteauthor{Diego}
