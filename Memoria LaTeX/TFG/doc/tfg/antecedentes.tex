\setcounter{chapter}{2}
\chapter{Antecedentes}
\label{chap:antecedentes}

\drop{H}{oy} en día existen numerosas aplicaciones para comunicarse de manera instantánea con otros usuarios, permitiendo la coordinación y organización de un grupo de trabajo o, simplemente, para hablar con los amigos. Esto último es lo que se conoce habitualmente como <<chatear>>. En este capítulo se presentará una serie de herramientas de mensajería instantánea junto con las características más destacadas que ofrecen a los usuarios finales y las que ofrecen a los desarrolladores, por ejemplo, si poseen una \acf{API}, si son multiplataforma o si se trata de aplicaciones de código abierto.

\section{WhatsApp}
\label{sec:whatsapp}

\begin{figure}
	\begin{center}
		\includegraphics[width=0.5\pagetotal]{/whatsapp_logo.jpg}
		\caption{Imagotipo de WhatsApp}
		\label{fig:whatsapp}
	\end{center}
\end{figure}

WhatsApp (Figura \ref{fig:whatsapp}) es una de las aplicaciones de mensajería instantánea más usadas a nivel de usuario, probablemente por el hecho de que fue una de las primeras en llegar a los \textit{smartphones} o teléfonos inteligentes. En sus comienzos, la aplicación venía a suplir la carencia que acusaba Jan Koum, uno de los creadores, de poder ver un estado al lado de cada uno de los contactos de la agenda de, por aquel entonces, un recién lanzado iPhone. De esta manera, un usuario podría conocer qué estaba haciendo un contacto mediante un texto corto o <<estado>>. Finalmente, el 24 de febrero de 2009, WhatsApp se creó como empresa y producto y a principios de 2011 ya se encontraba en el top 20 de aplicaciones en la \textit{App Store} de Apple en Estados Unidos.

\newpage

El 19 de febrero de 2014 la empresa de Mark Zuckerberg, Facebook, anunció la compra de WhatsApp por 19.000 millones de dólares, aunque no sería hasta octubre del mismo año cuando se llevara a cabo la compra definitiva por 21.800 millones de dólares \cite{Novoa2014}.

\subsection{Funcionalidades para los Usuarios}
Aunque su principal función fue la de ver el estado de los contactos como hacía en sus inicios, WhatsApp fue evolucionando y convirtiéndose en una aplicación de mensajería instantánea. Además de permitir mantener una comunicación en tiempo real, algunas de sus características principales a nivel de usuario son las siguientes \cite{WhatsApp2017}:

\begin{itemize}
	\item Grupos de chat de hasta 256 personas para compartir mensajes, fotos y vídeos con la posibilidad de asignarles un nombre, silenciarlos, personalizar notificaciones, etc. Cada uno de estos grupos puede tener uno o varios administradores.
	\item Llamadas y videollamadas mediante \acf{VoIP}.
	\item Posibilidad de continuar una conversación en un navegador web u ordenador personal mediante \textit{WhatsApp Web} o la aplicación de escritorio. En este caso, el teléfono ha de estar conectado a Internet puesto que hace de <<puente>>, reenviando los mensajes al ordenador.
	\item Cifrado de extremo a extremo mediante el protocolo <<Signal>>.
	\item Posibilidad de enviar archivos de hasta 100 \acs{MB}.
	\item Envío de mensajes de voz.
\end{itemize}

%TODO: Mirar si metemos referencias en los sistemas operativos móviles

\subsection{Funcionalidades para los Desarrolladores}
Desgraciadamente, WhatsApp no cuenta con una \acs{API} abierta que pueda ser utilizada por los desarrolladores para crear otras aplicaciones basadas en la original. A pesar de ello, es multiplataforma y se encuentra disponible en diferentes sistemas operativos (además de poder usarse desde un navegador web): Android, iOS, Windows Phone, Nokia Symbian S40, BlackBerry, Windows y macOS. El número de usuarios que pueden participar en un único grupo, por el momento, es de 256. Además, cada cuenta de WhatsApp ha de estar vinculada obligatoriamente a un número de teléfono móvil.

\clearpage

\section{Telegram}
\label{sec:telegram}

Telegram (Figura \ref{fig:telegram}) llegó con varios años de retraso con respecto a WhatsApp, su principal rival, que se encontraba ya en una posición, en cierto modo, cómoda y consolidada. Se lanzó en el año 2013, aunque no sería traducido al español y llevado a los ordenadores hasta pasado un año, en 2014. Esta es una de las principales consecuencias de que Telegram no acapare una cuota de usuarios tan grande, pues actualmente tiene cien millones de usuarios frente a los mil millones que tiene WhatsApp \cite{Ramirez2017}. No obstante, esta aplicación ofrece muchas características que WhatsApp no tiene, al menos, por ahora.

\begin{figure}[!h]
	\begin{center}
		\includegraphics[width=0.16\textwidth]{/telegram_logo.png}
		\caption{Isotipo de Telegram}
		\label{fig:telegram}
	\end{center}
\end{figure}

\subsection{Funcionalidades para los Usuarios}
Telegram posee un rango de características algo más amplio, siendo éstas las más destacadas \cite{Telegram2017}:

\begin{itemize}
	\item Coordinar grupos de chat de hasta 10.000 miembros (<<supergrupos>>).
	\item Existen también los <<canales>>, donde los usuarios pueden unirse e interactuar en un modo de sólo lectura, no pudiendo enviar ningún tipo de mensaje o archivo.
	\item Los historiales y conversaciones se almacenan en un servidor externo, no dependiendo del dispositivo donde se esté ejecutando la aplicación.
	\item Posibilidad de enviar archivos de hasta 1,5 \acf{GB}.
	\item Cifrado de mensajes mediante el protocolo propietario de Telegram: \textit{MTProto}.
	\item Borrado de mensajes no sólo en el dispositivo de origen, sino también en el de destino dentro de una conversación.
	\item Capacidad de hacer las veces de <<nube personal>> ilimitada, al poder enviarse a uno mismo los archivos, mensajes o información que desee conservar.
	\item Llamadas mediante \acs{VoIP}.
	\item Existencia de chats secretos. Esta modalidad consiste en que se usa un cifrado de extremo a extremo, a diferencia del resto de chats, que usan cifrado cliente-servidor/servidor-cliente \cite{Telegram2017a}. Además, estas conversaciones no se guardan en los servidores de Telegram y se puede establecer un contador para fijar el tiempo tras el cual cada mensaje enviado será borrado.
\end{itemize}

\subsection{Funcionalidades para los Desarrolladores}

\begin{itemize}
	\item Se trata de un software \textit{Open Source}. Esto quiere decir que el código del cliente se encuentra disponible, aunque no pasa lo mismo con el código del servidor.
	\item Existen dos tipos de \acs{API}: una para desarrollar aplicaciones basadas en Telegram y otra para la creación de \textit{Bots}, que se podrían definir como asistentes a los que se pueden realizar consultas dentro de la aplicación.
	\item Es un servicio multiplataforma, disponible en los siguientes sistemas operativos (además de tener una versión web): Android, iOS, Firefox OS, Windows, macOS y Linux.
\end{itemize}

\section{Slack}
\label{sec:slack}

Slack (Figura \ref{fig:slack}) es una aplicación destinada a los conjuntos de personas que trabajan sobre un mismo tema, encontrándose enfocado, principalmente, a equipos y grupos de trabajo. Ha introducido un concepto algo diferente en el ámbito de la mensajería instantánea, mejorando el conocido correo electrónico, puesto que se pueden mantener conversaciones privadas, crear canales públicos o compartir archivos. Originalmente se trataba de una herramienta interna que se comenzó a usar para el desarrollo de un juego en \textit{flash}, llamado \textit{Glitch}. Más tarde, Stewart Butterfield, su fundador, confesó que fracasaron al realizar el juego. A finales de 2012, Butterfield comunicó en \textit{Twitter} que no continuarían con el juego, aunque la empresa, \textit{Tiny Speck}, seguiría <<viva>> \cite{Thomas2015}. Slack se lanzó en febrero de 2014 con alrededor de 15.000 usuarios, para finales de ese año eran 285.000 y la cifra ha ido aumentando hasta los más de tres millones que posee actualmente \cite{PyMEs2017}.

\begin{figure}[!h]
	\begin{center}
		\includegraphics[width=0.3\textwidth]{/slack_logo.png}
		\caption{Imagotipo de Slack}
		\label{fig:slack}
	\end{center}
\end{figure}

Esta aplicación integra de manera eficiente servicios adicionales como Google Drive, GitHub, Dropbox o Google Hangouts, entre otros. La comunicación se realiza mediante <<canales>> a los que los usuarios pueden entrar para hablar con otros, compartir archivos y documentos o enlaces.

\clearpage

Slack, aunque puede usarse de manera gratuita, posee una versión de pago si se desea disponer de funcionalidades adicionales o de una versión más completa, siendo estas las diferentes formas de tarificación:

\begin{itemize}
	\item \textbf{\textit{Free}}. Para equipos pequeños y de uso por tiempo ilimitado. Como restricciones tiene que sólo se pueden buscar mensajes entre los 10.000 últimos, integración con un máximo de 10 aplicaciones de terceros o videollamadas de uno a uno. \textbf{Precio:} gratuito.
	\item \textbf{\textit{Standard}}. Ofrece más ventajas como integración ilimitada con aplicaciones de terceros, videoconferencias de hasta 15 personas o 10 \acs{GB} por miembro de equipo. \textbf{Precio:} 6,25\euro{} por usuario, al mes, si se contrata anualmente.
	\item \textbf{\textit{Plus}}. Ofrece todas las características disponibles de Slack, así como una disponibilidad del servicio muy alta o 20 \acs{GB}. \textbf{Precio:} 11,75\euro{} por usuario, al mes, si se contrata anualmente.
\end{itemize}

\subsection{Funcionalidades para los Usuarios}
Las características de Slack están más enfocadas al uso corporativo, siendo estas las más destacadas \cite{Slack2017}:

\begin{itemize}
	\item Creación de canales para conversar entre personas del mismo grupo de trabajo, siendo un concepto similar al de <<grupo>> en WhatsApp o Telegram, pudiendo ser canales públicos o privados. Los canales públicos son canales para proyectos, grupos y temas visibles para toda la organización. Los mensajes en estos canales se archivan y se pueden recuperar mediante búsquedas. En cuanto a los canales privados, están destinados a temas más sensibles, solo se puede entrar a ellos mediante invitación y los mensajes únicamente son visibles para los miembros.

	\item Llamadas y videollamadas integradas con posibilidad de compartir la pantalla.
	\item Mensajes directos.
	\item Posibilidad de crear <<cuentas de invitado>>, para dar acceso a la misma información.
	\item Posibilidad de enviar archivos a un canal, a los que los demás usuarios pueden hacer comentarios y hacer búsquedas avanzadas para encontrarlos.
	\item Recomendación de canales.
	\item Integración con otros servicios como Google Drive, Dropbox, GitHub o Google Duo.
\end{itemize}

\subsection{Funcionalidades para los Desarrolladores}
Al igual que sucede con Telegram, Slack también cuenta con una \acs{API}. No obstante, se trata de una aplicación que no es \textit{Open Source}. Por último, los sistemas operativos en los que Slack puede instalarse y usarse son los siguientes: Android, iOS, Windows Phone, Windows, macOS y Linux.

\newpage

\section{Skype}
\label{sec:skype}

Skype (Figura \ref{fig:skype}) es una de las aplicaciones de mensajería, llamadas y videollamadas más conocidas en el entorno del ordenador personal.
El 29 de agosto de 2003 se lanzaba la primera beta, tratándose de un cliente \acf{P2P} gratuito cuyo código fuente no era abierto. Más tarde, en 2005, fue adquirida por eBay por 2.600 millones de dólares \cite{Velasco2013}. Posteriormente, en mayo de 2011, se anuncia la compra de Skype por parte de Microsoft por la cifra de 8.500 millones de dólares y, en noviembre de 2012, sería esta aplicación la que sustituiría al conocido \textit{Messenger} \cite{Ramirez2013}.

\begin{figure}[!h]
	\begin{center}
		\includegraphics[width=0.35\textwidth]{/skype_logo.png}
		\caption{Isologo de Skype}
		\label{fig:skype}
	\end{center}
\end{figure}

\subsection{Funcionalidades para los Usuarios}
Skype, a diferencia de las aplicaciones presentadas anteriormente, tiene unas características más enfocadas a las llamadas, como \cite{Skype2017}:

\begin{itemize}
	\item Llamadas y videollamadas gratuitas, individuales y grupales, entre usuarios de Skype.
	\item Llamadas a teléfonos fijos y móviles con cierto coste, que varía, principalmente, en función del país al que se llama.
	\item Los contactos de un usuario pueden llamar a un teléfono, recibiendo el destinatario la llamada en Skype. Esto es lo que se conoce como <<Número de Skype>>.
	\item Desvío de llamadas a cualquier teléfono.
	\item \textit{Skype To Go}: posibilidad de llamar a números internacionales desde cualquier teléfono con un coste añadido.
	\item Posibilidad de usar un chat de mensajería instantánea, individual y grupal, donde, además de enviar texto, se pueden enviar archivos de cualquier tamaño, \acs{SMS} o mensajes de voz.
	\item Capacidad para poder compartir pantalla de manera individual y grupal.
	\item \textit{Skype Translator}: traducción de llamadas, videollamadas y mensajes instantáneos en tiempo real.
\end{itemize}

\newpage

\subsection{Funcionalidades para los Desarrolladores}
En la página web de Skype únicamente se ofrece una limitada cantidad de posibilidades \cite{Skype2017a}:

\begin{itemize}
	\item Creación de \textit{bots}, con los que se puede interactuar mediante chat, voz o vídeo.
	\item Posibilidad de integrar videollamadas y chat en una página web.
	\item Pagos integrados.
	\item Integrar aplicaciones, como YouTube o Giphy, para mandar vídeos e imágenes en movimiento sin tener que abandonar la aplicación principal (\textit{Add-ins}).
\end{itemize}

\section{Signal}
\label{sec:signal}

Signal (Figura \ref{fig:signal}) es la aplicación de mensajería conocida por el llamado <<Caso Snowden>>, en el que Edward Snowden filtró los casos de espionaje de la \acf{NSA}. El propio Snowden fue el que recomendó esta aplicación mediante un \textit{tweet} (Figura \ref{fig:tweetsnowden}) en su cuenta de Twitter debido a la gran seguridad y privacidad que ofrecía su método de encriptación, cifrando de punto a punto las conversaciones. Más tarde, WhatsApp adoptaría este método de cifrado, incluyéndolo por defecto en todas las conversaciones de su aplicación. A diferencia de WhatsApp, Signal no guarda ningún metadato en los servidores, como el tiempo de conversación, quién habla con quién o cuándo lo hace.

\begin{figure}[!h]
	\begin{center}
		\includegraphics[width=0.2\textwidth]{/signal_logo.png}
		\caption{Isotipo de Signal}
		\label{fig:signal}
	\end{center}
\end{figure}

\begin{figure}[!h]
	\begin{center}
		\includegraphics[width=0.7\textwidth]{/tweet_snowden}
		\caption{\textit{Tweet} de Snowden sobre Signal}
		\label{fig:tweetsnowden}
	\end{center}
\end{figure}

\newpage

\subsection{Funcionalidades para los Usuarios}
Signal posee unas características algo más reducidas que las anteriores alternativas, pero esto se debe a que está enfocada a la seguridad y privacidad, como se ha descrito anteriormente \cite{Signal2017}.

\begin{itemize}
	\item Conversaciones individuales y grupales, en las que se puede enviar texto, mensajes de voz, vídeo, documentos e imágenes.
	\item Llamadas de voz y de vídeo.
	\item Encriptación de mensajes punto a punto.
	\item Posibilidad de programar un contador para hacer desaparecer los mensajes enviados.
\end{itemize}

\subsection{Funcionalidades para los Desarrolladores}
Al contrario que sucede con WhatsApp, Signal es de código abierto, por lo que el código se encuentra disponible en su página de GitHub para que pueda ser revisado por la comunidad. Las características más destacadas son que ofrece el protocolo de la \acs{API} y se encuentra disponible en los siguientes sistemas operativos: Android, iOS mediante aplicación nativa y Windows, macOS, Linux y ChromeOS mediante una aplicación para el navegador web Google Chrome.

\section{Wickr}
\label{sec:wickr}

Wickr fue lanzada en junio de 2012 originalmente sólo para iOS, el sistema operativo de los dispositivos de Apple. No obstante, con el tiempo fueron apareciendo más versiones para otros sistemas operativos. Últimamente ha tenido una mayor repercusión debido a que aparece en una serie de televisión: \textit{Mr. Robot} \cite{Elio2016}. Al igual que Signal, su principal premisa es la seguridad y privacidad de las conversaciones de los usuarios, siendo su diseño bastante austero. No es necesario registrar el número de teléfono o una dirección de correo electrónico para comunicarse, al menos, en \textit{Wickr Me}. Además, no se guarda metadato alguno. Posee diferentes modalidades de tarificación diferenciándose principalmente en el peso máximo de los archivos, el tiempo de expiración de los mensajes y controles administrativos \cite{Wickr2017}:

\begin{itemize}
	\item \textbf{\textit{Wickr Me}}. \textbf{Precio}: gratuito.
	
	\begin{itemize}
		\item Tiempo máximo de desaparición de mensajes de seis días.
		\item Tamaño máximo de los archivos de 10 \acs{MB}.
		\item No dispone de \textit{Secure Rooms} (equipos y proyectos de hasta 50 usuarios), controles administrativos o chat de voz y vídeo.		
	\end{itemize}

	\newpage

	\item \textbf{\textit{Wickr Plus}}. \textbf{Precio}: 15\$ por usuario al mes, unos 13\euro{}.
	
	\begin{itemize}
		\item Tiempo máximo de desaparición de mensajes de seis días.
		\item Tamaño máximo de los archivos de 1 \acs{GB}.
		\item Dispone de algunos controles administrativos.
	\end{itemize}

	\item \textbf{\textit{Wickr Enterprise}}. \textbf{Precio}: se debe contactar con el departamento de ventas.

	\begin{itemize}
		\item Tiempo máximo de desaparición de mensajes de un año.
		\item Tamaño máximo de los archivos de 5 \acs{GB}.
		\item Dispone de todos los controles administrativos.
	\end{itemize}

	\item \textbf{\textit{Wickr Pro}}. \textbf{Precio}: 25\$ por usuario al mes, unos 21\euro{}.

	\begin{itemize}
		\item Tiempo máximo de desaparición de mensajes de un año.
		\item Tamaño máximo de los archivos de 5 \acs{GB}.
		\item Dispone de algunos controles administrativos (equilibrio entre \textit{Wickr Plus} y \textit{Wickr Enterprise}).
	\end{itemize}

\end{itemize}

\begin{figure}[!h]
	\begin{center}
		\includegraphics[width=0.4\textwidth]{/wickr_logo}
		\caption{Imagotipo de Wickr}
		\label{fig:wickr}
	\end{center}
\end{figure}

\subsection{Funcionalidades para los Usuarios}
Las principales características que Wickr ofrece a sus usuarios son:

\begin{itemize}
	\item Encriptación punto a punto.
	\item Autodestrucción de mensajes.
	\item Posibilidad de colaboración dentro de un equipo de trabajo.
	\item Envío de archivos de hasta 5 \acs{GB}, tal y como se ha detallado en las modalidades de tarificación.
\end{itemize}

\subsection{Funcionalidades para los Desarrolladores}
Desafortunadamente, Wickr no tiene disponible más que una implementación en C del protocolo de envío de mensajes en su página de \textit{GitHub}. Wickr se encuentra disponible en los siguientes sistemas operativos: Android, iOS, Windows, macOS y Ubuntu.

\newpage

\section{Comparación de Alternativas}
Finalmente, en la Tabla \ref{tab:whatsappvstelegram}, se van a comparar las dos alternativas más populares, mostrando y enfrentando sus funcionalidades más destacadas.

\begin{table}[hp]
	\centering
	{\small
		\begin{tabular}{p{.3\textwidth}p{.3\textwidth}p{.3\textwidth}}
	\tabheadformat
	                     &
	\tabhead{WhatsApp}   &
	\tabhead{Telegram}   \\
	\hline
	\textbf{<<Chats>> grupales} & Hasta 256 personas & Hasta 10.000 personas \\
	\hline
	\textbf{Llamadas integradas} & Sí & Sí \\
	\hline
	\textbf{Videollamadas integradas} & Sí & No \\
	\hline
	\textbf{Posee aplicación de escritorio} & Sí (requiere de teléfono móvil) & Sí (conversaciones en la nube) \\
	\hline
	\textbf{Cifrado} & Signal (de extremo a extremo por defecto en todas las conversaciones) & De extremo a extremo en chats secretos, servidor-cliente en el resto de conversaciones \\
	\hline
	\textbf{Envío de archivos multimedia} & Sí & Sí \\
	\hline
	\textbf{Envío de documentos y otros archivos} & Sí, con límite de 100 \acs{MB} & Sí, hasta 1,5 \acs{GB} y sin extensión específica. \\
	\hline
	\textbf{Mensajes de voz} & Sí & Sí \\
	\hline
	\textbf{Canales} & No & Sí \\
	\hline
	\textbf{Respaldo de <<chats>> en la nube} & Sí (en la versión de escritorio dependen del móvil) & Sí (en los servidores de Telegram) \\
	\hline
	\textbf{<<Chats>> secretos} & No & Sí \\
	\hline
	\textbf{Creación de \textit{bots}} & No & Sí \\
	\hline
	\textbf{Open Source} & No & Sí \\
	\hline
	\textbf{Existencia de una API} & No & Sí \\
	\hline
	\textbf{Multiplataforma} & Sí & Sí \\
	\hline
\end{tabular}

% Local variables:
%   coding: utf-8
%   ispell-local-dictionary: "castellano8"
%   TeX-master: "main.tex"
% End:
	}
	\caption[WhatsApp vs Telegram]
	{WhatsApp vs Telegram}
	\label{tab:whatsappvstelegram}
\end{table}

\newpage

\section{Aplicaciones Especializadas}
En esta sección se presentan ciertas aplicaciones que, a diferencia de las mencionadas anteriormente, han sido específicamente desarrolladas para su uso en el sector educativo.

\subsection{miColegioApp}
Esta aplicación ofrece un canal de comunicación directo, oficial, inmediato y seguro entre centros escolares y familias. Para los colegios tiene una serie de ventajas como enviar circulares, mensajes personalizados, fotos o documentos \acs{PDF}; conocer estadísticas en tiempo real sobre la lectura de notificaciones o la aplicación de filtros en el envío: por alumno, grupo, curso, actividad, etc. Para los padres tiene otras funcionalidades como confirmar citas o firmar autorizaciones. Está disponible para dispositivos Android e iOS y cuenta con una versión web \cite{creaTactil}.

\begin{figure}[!h]
	\begin{center}
		\includegraphics[width=0.16\textwidth]{/micolegioapp_logo.png}
		\caption{Isologo de miColegioApp}
		\label{fig:micolegioapp}
	\end{center}
\end{figure}

\subsection{BabyNotez}
<<BabyNotez>> (Figura \ref{fig:babynotez}) está pensada para los centros de educación infantil. Su finalidad principal es la de sustituir, aunque también añadir, características de la clásica agenda contando, por ejemplo, con mensajería instantánea bidireccional. Cada día los profesores redactarán un pequeño informe que los padres recibirán con una notificación sobre cómo ha acontecido el día del alumno. Se trata de una aplicación gratuita para las familias, pero con un coste no determinado para el centro por la activación de la intranet, puesto que cada uno requiere de una pequeña red interna que gestiona la empresa responsable de la aplicación, necesaria para su funcionamiento. Se encuentra disponible para dispositivos iOS y Android \cite{Educo2016}.

\begin{figure}[!h]
	\begin{center}
		\includegraphics[width=0.3\textwidth]{/babynotez_logo.png}
		\caption{Isologo de BabyNotez}
		\label{fig:babynotez}
	\end{center}
\end{figure}

\newpage

\subsection{TokApp School}
Esta plataforma (Figura \ref{fig:tokapp}) proporciona una plataforma de gestión online desde la que el centro realiza todas las tareas de tipo organizacional, como dar de alta y gestionar alumnos y clases. A los padres y alumnos les permite enviar mensajes y notificaciones, además de comunicarse con el centro y otros estudiantes. Al profesorado le evita el envío de circulares o documentos en papel puesto que la aplicación permite también el envío de archivos adjuntos. Es gratuita para los padres pero para los centros tiene un coste que no está especificado en su página Web \cite{Educo2016}.

\begin{figure}[!h]
	\begin{center}
		\includegraphics[width=0.25\textwidth]{/tokapp_logo.jpg}
		\caption{Isotipo de TokApp School}
		\label{fig:tokapp}
	\end{center}
\end{figure}

% Local Variables:
%  coding: utf-8
%  mode: latex
%  mode: flyspell
%  ispell-local-dictionary: "castellano8"
% End: