\chapter{Objetivos}
\label{chap:objetivos}

\noindent

%Para este capítulo, la normativa indica:
%
%«Concretar y exponer el problema a resolver describiendo el entorno de trabajo,
%la situación y detalladamente qué se pretende obtener. Limitaciones y
%condicionantes a considerar para la resolución del problema (lenguaje de
%construcción, equipo físico, equipo lógico de base y de apoyo, etc.). Si se
%considera necesario, esta sección puede titularse ``Objetivos e hipótesis de
%trabajo''. En este caso, se añadirán las hipótesis de trabajo que el alumno, con
%su TFG, pretende demostrar».

\drop{A}{lo} largo de este capítulo se expondrán tanto el objetivo principal como los diferentes objetivos específicos del Trabajo Fin de Grado, así como las posibles limitaciones o condicionantes que éstos pudieran conllevar.

\section{Objetivo general}

%El principal objetivo de este trabajo es el desarrollo de una aplicación móvil para el sistema operativo \textit{Android}, orientada, principalmente, al sector docente. Esta aplicación se apoyará en un \textit{backend} suministrado por la empresa Google, llamado \textit{Firebase}. Para su desarrollo se utilizará el \acf{IDE}, o Entorno de Desarrollo Integrado, en español, \textit{Android Studio}.

El principal objetivo de este trabajo es implementar una herramienta de mensajería instantánea para la comunicación entre el profesorado y los padres de los alumnos específica para el contexto educativo. Se apoyará en un \textit{backend} suministrado por la empresa Google, llamado \textit{Firebase}. Para su desarrollo se utilizará el \acf{IDE}, o Entorno de Desarrollo Integrado, en español, \textit{Android Studio}.

La motivación por la que surge esta idea reside en la necesidad de disponer de una aplicación adaptada y acotada a las necesidades primordiales en la gestión de la comunicación con los padres por parte del centro educativo. Actualmente, la mayor parte de las alternativas que existen son más generalistas, orientándose a cualquier tipo de público. Por tanto, éstas incluyen funciones cuya utilidad, en el sector profesional de la docencia sería, en determinados casos, cuestionable.

Adicionalmente, en la actualidad, existen diversos problemas sociales con los grupos de mensajería entre los padres y con los profesores. Estas situaciones favorecen también las intenciones de realizar una aplicación que trate de eliminarlos o, al menos, evitarlos.

%El hito final que se pretende lograr, destacando el problema específico que
%resuelve o la funcionalidad que aporta la aplicación o sistema desarrollado.

\section{Objetivos específicos}

En esta sección se detallarán los objetivos específicos a completar para cumplir el objetivo general.

%Los objetivos específicos (o sub-objetivos) son las partes independientes del
%proyecto que tienen valor para el cliente por si mismas. Es habitual confundir
%los objetivos específicos con requisitos, fases o tareas. Para aclarar la
%diferencia pongo aquí un ejemplo.
%
%Supón que el objetivo general es destruir una flota enemiga. Los
%objetivos específicos podrían ser: hundir el portaaviones, inutilizar las
%torretas de los destructores, eliminar los cazas enemigos, etc.
%
%Los sub-objetivos \textbf{no son tareas ni fases}.  En nuestro ejemplo, las
%tareas podrían ser: determinar las fuerzas de la flota enemiga, elegir el
%armamento más adecuado, hacer la dotación de los buques propios. Es importante
%tener claro que las tareas no tienen valor intrínseco para el cliente. Por
%ejemplo, si en un proyecto con desarrollo en cascada se cancela en la fase de
%diseño, no se le entrega nada de valor al cliente, es decir, el análisis
%realizado no cubre ningún sub-objetivo.
%
%Los sub-objetivos \textbf{no son requisitos}. Los requisitos son concrecciones o
%limitaciones que determina el cliente. En nuestro ejemplo bélico,algunos
%requisitos podrían ser: eliminar primero los cazas enemigos, utilizar únicamente
%explosivos convencionales.
%
%Tampoco se deben confundir los objetivos del proyecto con los objetivos del
%alumno. Indicar como objetivo que el alumno va a aprender o a estudiar
%determinada disciplina o herramienta no aporta nada al cliente. Deben ser
%entregables que el cliente puede valorar y por los que estaría dispuesto a
%pagar. Resumiendo, tienen que ser \textbf{objetivos}, no subjetivos.

\subsection{Objetivo I: Gestión de los Usuarios}

Este objetivo será de especial relevancia puesto que se deberá poder realizar una gestión sencilla y efectiva de los diferentes usuarios de la aplicación final por parte del centro o personal docente.

\newpage

\subsection{Objetivo II: Establecimiento de diferentes roles}

En este caso, se deberán elegir y fijar diferentes roles, que serán asignados a las personas que utilicen la aplicación. Por ejemplo, el rol de moderado que será destinado, principalmente, a los tutores, profesores o personal del centro que use la aplicación.

\subsection{Objetivo III: Establecimiento de políticas para el uso}

Estas políticas definirán el correcto uso dentro del ámbito educativo, sector en el que, actualmente, ocurren diversos malentendidos y surgen problemas relacionados con las aplicaciones de mensajería instantánea.

\subsection{Objetivo IV: Interacción entre PC y dispositivo móvil}

La interacción entre estos dos dispositivos será importante puesto que, desde el PC se podrá controlar la asignación de ciertos usuarios a determinados grupos, así como su borrado de los mismos. Además, se podrán realizar diversas tareas de administración.

\subsection{Objetivo V: Integración de la aplicación con otros servicios}

Este objetivo se centrará en el estudio de cómo compenetrar la aplicación con otros servicios, como \textit{Google Calendar}, de manera que se puedan agregar nuevos eventos de calendario.

%\subsection{Objetivo II: Estudio de la plataforma}
%
%Tras realizar el estudio de las diversas soluciones y de haber tomado una decisión, se procederá al estudio de la plataforma para extraer todo el potencial que pueda proporcionar a la hora del desarrollo. De esta manera, todas las funcionalidades que ofrezca podrán ser aprovechadas para que la aplicación pueda tener el máximo nivel de calidad posible para los usuarios finales.
%
%\subsection{Objetivo III: Estudio de los requisitos}
%
%En este objetivo se tratará de especificar los requisitos que debe cumplir la implementación final para ofrecer a los usuarios de la misma.
%
%\subsection{Objetivo IV: Integración de la aplicación con otros servicios}
%
%Este objetivo se centrará en el estudio de cómo compenetrar la aplicación con otros servicios, como \textit{Google Calendar}, de manera que se puedan agregar nuevos eventos de calendario.


% Local Variables:
%  coding: utf-8
%  mode: latex
%  mode: flyspell
%  ispell-local-dictionary: "castellano8"
% End:
