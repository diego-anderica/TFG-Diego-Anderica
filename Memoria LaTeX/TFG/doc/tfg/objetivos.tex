\chapter{Objetivos}
\label{chap:objetivos}

\noindent

\drop{E}{n} este capítulo se expondrán tanto el objetivo principal como los diferentes objetivos específicos del Trabajo Fin de Grado, así como las posibles limitaciones o condicionantes que se pudieran derivar de los mismos.

\section{Objetivo general}
El principal objetivo de este trabajo es implementar una herramienta de mensajería instantánea para la comunicación entre el profesorado y los padres de los alumnos específica para el contexto educativo. Se apoyará en un \textit{backend} proporcionado por la empresa Google, llamado \textit{Firebase}. Para su desarrollo se utilizará el Entorno de Desarrollo Integrado \textit{Android Studio}. La motivación por la que surge esta idea reside en la necesidad de disponer de una aplicación adaptada y acotada a las necesidades primordiales en la gestión de la comunicación con los padres por parte del centro educativo. Actualmente, la mayor parte de las alternativas que existen son más generalistas, orientándose a cualquier tipo de público. Por tanto, éstas incluyen funciones cuya utilidad en el sector profesional de la docencia sería, en determinados casos, cuestionable. Adicionalmente, en la actualidad, los grupos de mensajería pueden generar diversos problemas de convivencia. Por todo lo anteriormente mencionado, se considera de interés realizar una aplicación que trate de minimizar la ocurrencia de situaciones no deseadas.

\section{Objetivos específicos}
En esta sección se detallarán los objetivos específicos a completar para poder así cumplir el objetivo general.

\newpage

\subsection{Objetivo I: Implementar un marco de gestión de usuarios vinculado al contexto educativo}
Se pretende implementar un mecanismo de gestión de usuarios que sea lo más eficiente y sencillo de utilizar. Por ejemplo, existe la posibilidad de que el personal del centro realice importación de archivos .csv para la creación de los grupos de chat. De igual manera, se deberán elegir y fijar diferentes roles, que serán asignados a las personas que utilicen la aplicación: el rol de moderador que estará destinado, principalmente, a los tutores, profesores o personal del centro que use la aplicación y que será el encargado de validar finalmente el envío de mensajes y el rol de usuario normal, destinado a los padres que se encuentren registrados.

\subsection{Objetivo II: Proporcionar un entorno de ejecución multiplataforma}
Puesto que el personal del centro tendrá las funciones de importar datos, registrar a los usuarios, crear y mantener los grupos de chat, éstas se realizarán mediante el uso de un ordenador personal puesto que, de esta manera, dichas tareas se tornan más sencillas de realizar y de un \textit{frontend} que resulte amigable y fácil de utilizar por parte del profesorado y personal del centro. Por otra parte, el resto de los usuarios de la aplicación accederán a la misma mediante el teléfono móvil o correo electrónico, puesto que los usuarios podrán elegir el método de entrada.

\subsection{Objetivo III: Implementar un mecanismo de monitorización activa del tipo y contenido de los mensajes}
Se busca disponer de algún tipo de monitorización para detectar mensajes que puedan ser potencialmente improcedentes dentro del contexto educativo. Esto se puede conseguir mediante el uso de la plataforma \textit{BlueMix} de IBM, desde que la que se podrá acceder a <<Watson>>, un sistema de inteligencia artificial con el que se pueden controlar los mensajes que son enviados a través de la aplicación. Una vez enviado el texto a analizar, éste debe ser interpretado por la aplicación, puesto que es devuelto en un formato diferente. Además, se dispondrá de cierta moderación puesto que el profesor responsable de cada grupo tendrá la capacidad de validar los mensajes que el resto de usuarios envíen a dicho grupo.

\subsection{Objetivo IV: Integración de la aplicación con Google Calendar}
Este objetivo se centrará en el estudio de cómo compenetrar la aplicación con otros servicios, como \textit{Google Calendar}, de manera que se puedan agregar nuevos eventos de calendario sin que el usuario tenga que cambiar de aplicación manualmente. Ejemplos de estos eventos podrían ser añadir nuevos exámenes, reuniones, tutorías con los profesores, etc.

%TODO: Comunicación privada?
\subsection{Objetivo V: Implementar mecanismos que permitan comunicaciones personales profesor-tutores del alumno y viceversa}
Además de la comunicación mediante los grupos creados por el personal del centro, los usuarios de la aplicación podrán iniciar una conversación con los profesores mediante la creación de un chat privado. Del mismo modo, los profesores podrán crear chats privados con el resto de usuarios.

% Local Variables:
%  coding: utf-8
%  mode: latex
%  mode: flyspell
%  ispell-local-dictionary: "castellano8"
% End: