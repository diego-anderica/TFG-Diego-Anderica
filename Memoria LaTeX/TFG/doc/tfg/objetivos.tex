\chapter{Objetivos}
\label{chap:objetivos}

\noindent

\drop{E}{n} este capítulo se expondrán tanto el objetivo principal como los diferentes objetivos específicos del \acs{TFG}, así como las posibles limitaciones o condicionantes que se pudieran derivar de los mismos.

\section{Objetivo General}
El principal objetivo de este trabajo se centra en implementar una herramienta de mensajería instantánea específica para el contexto educativo que permita la comunicación entre el profesorado y los padres o tutores legales de los alumnos. La motivación principal del \acs{TFG} reside en la necesidad de disponer de una aplicación adaptada y acotada para la comunicación con los padres desde el centro educativo. Actualmente, la mayor parte de las alternativas que existen son generalistas orientándose a cualquier tipo de contexto comunicativo. Por tanto, estas incluyen funcionalidades cuya utilidad en el sector profesional de la docencia sería, en determinados casos, cuestionable. Adicionalmente, se conoce que los grupos de mensajería pueden generar una serie de problemas de convivencia. Por tanto, teniendo en cuenta lo anterior, se considera de interés realizar una aplicación que trate de minimizar la ocurrencia de situaciones no deseadas.

\section{Objetivos Específicos}
En esta sección se detallarán los objetivos específicos que ayudarán a la consecución del objetivo general.

\clearpage

\subsection{Objetivo I: Implementar un marco de gestión de usuarios vinculado al contexto educativo}
Se pretende implementar un mecanismo de gestión de usuarios que sea eficiente y sencillo de utilizar. Por ejemplo, existe la posibilidad de que el personal del centro realice importación de archivos \acs{CSV} mediante una página Web, volcando automáticamente los datos en una base de datos para la posterior creación de chats desde la aplicación móvil. De igual manera, se deberán elegir y fijar determinados roles que serán asumidos por las diferentes personas que utilicen la aplicación: el rol de administrador de grupo, que estará destinado, principalmente, a los tutores, profesores o personal del centro que use la aplicación y  el rol de usuario normal, destinado a los tutores legales que se encuentren registrados. Además, deberá haber una persona que asuma el rol de administrador del sistema, encargada del tratamiento de los datos desde la página Web y de agregar nuevos administradores del sistema.

\subsection{Objetivo II: Proporcionar un entorno de ejecución multiplataforma}
Puesto que el administrador del sistema tendrá que desempeñar tareas como la de importar datos de las familias o registrar a los usuarios, estas se realizarán mediante el uso de un ordenador personal y, para ello, se diseñará un \textit{frontend} que resulte amigable y fácil de utilizar. Por otra parte, el resto de los usuarios accederán al sistema mediante la aplicación móvil, usando el número de teléfono o correo electrónico para identificarse.

\subsection{Objetivo III: Implementar un mecanismo de monitorización activa del tipo y contenido de los mensajes}
Se desea disponer de un mecanismo de detección de mensajes que puedan ser potencialmente inadecuados dentro de un contexto educativo. Esto se puede conseguir mediante el uso de la plataforma \textit{Bluemix} de IBM, desde que la que se podrá acceder a <<Watson>>, un sistema de inteligencia artificial que ofrece múltiples posibilidades. Una de ellas es la de analizar los <<tonos>> en un determinado texto. Una vez analizado cada mensaje y devuelto el resultado del tono de este, debe ser interpretado por la aplicación y mostrado de una manera adecuada al administrador del chat mediante un código de colores (rojo, naranja, amarillo y verde, de menos a más deseable).

\subsection{Objetivo IV: Integración de la aplicación con Google Calendar}
Se integrará la aplicación con otros servicios como \textit{Google Calendar}, de manera que se puedan agregar nuevos eventos de calendario sin que el usuario tenga que cambiar de aplicación manualmente o realizar una inserción de información adicional. Ejemplos de estos eventos podrían ser añadir nuevos exámenes, reuniones, tutorías con los profesores, etc.

\subsection{Objetivo V: Implementar mecanismos que permitan comunicaciones privadas docente-tutores legales del alumno}
Además de la comunicación mediante los grupos creados por el personal del centro, se podrán crear <<salas>> de chat integradas por un docente y una familia. Es decir, un docente crearía un grupo de chat privado para comunicarse de manera exclusiva con una familia.

% Local Variables:
%  coding: utf-8
%  mode: latex
%  mode: flyspell
%  ispell-local-dictionary: "castellano8"
% End: