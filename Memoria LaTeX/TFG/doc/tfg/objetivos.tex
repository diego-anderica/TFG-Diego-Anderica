\chapter{Objetivos}
\label{chap:objetivos}

\noindent

\drop{E}{n} este capítulo se expondrán tanto el objetivo principal como los diferentes objetivos específicos del Trabajo Fin de Grado, así como las posibles limitaciones o condicionantes que se pudieran derivar de los mismos.

\section{Objetivo general}
El principal objetivo de este trabajo se centra en implementar una herramienta de mensajería instantánea específica para el contexto educativo que permita la comunicación entre el profesorado y los padres de los alumnos. Se apoyará en un \textit{backend} proporcionado por la empresa Google, llamado \textit{Firebase}. Para su desarrollo se utilizará el Entorno de Desarrollo Integrado \textit{Android Studio}. La motivación por la que surge esta idea reside en la necesidad de disponer de una aplicación adaptada y acotada a las necesidades primordiales en la gestión de la comunicación con los padres por parte del centro educativo. Actualmente, la mayor parte de las alternativas que existen son más generalistas, orientándose a cualquier tipo de público. Por tanto, estas incluyen funciones cuya utilidad en el sector profesional de la docencia sería, en determinados casos, cuestionable. Adicionalmente, hoy en día y como se ha en el capítulo anterior, los grupos de mensajería pueden generar diversos problemas de convivencia. Por tanto, teniendo en cuenta lo anteriormente mencionado, se considera de interés realizar una aplicación que trate de minimizar la ocurrencia de situaciones no deseadas.

\section{Objetivos específicos}
En esta sección se detallarán los objetivos específicos que ayudarán a la consecución del objetivo general.

\newpage

\subsection{Objetivo I: Implementar un marco de gestión de usuarios vinculado al contexto educativo}
Se pretende implementar un mecanismo de gestión de usuarios que sea eficiente y sencillo de utilizar. Por ejemplo, existe la posibilidad de que el personal del centro realice importación de archivos \acs{CSV} mediante una página Web, volcando automáticamente los datos en una base de datos para la posterior creación de los grupos de chat desde la aplicación móvil. De igual manera, se deberán elegir y fijar determinados roles que serán asumidos por las diferentes personas que utilicen la aplicación: el rol de administrador de grupo, que estará destinado, principalmente, a los tutores, profesores o personal del centro que use la aplicación y que podrá observar el tono de la conversación y de cada mensaje para actuar en consecuencia y el rol de usuario normal, destinado a los padres que se encuentren registrados. Además, deberá haber una persona que asuma el rol de administrador del sistema, encargada del tratamiento de los datos desde la página Web y de agregar nuevos administradores del sistema.

\subsection{Objetivo II: Proporcionar un entorno de ejecución multiplataforma}
Puesto que el administrador del sistema tendrá que desempeñar tareas como la de importar datos de las familias o registrar a los usuarios, estas se realizarán mediante el uso de un ordenador personal y de un \textit{frontend} que resulte amigable y fácil de utilizar. Por otra parte, el resto de los usuarios accederán al sistema mediante la aplicación móvil, usando el número de teléfono o correo electrónico para identificarse, puesto que se podrá elegir el método de entrada.

\subsection{Objetivo III: Implementar un mecanismo de monitorización activa del tipo y contenido de los mensajes}
Se busca disponer de alguna manera de detectar mensajes que puedan ser potencialmente improcedentes dentro del contexto educativo. Esto se puede conseguir mediante el uso de la plataforma \textit{Bluemix} de IBM, desde que la que se podrá acceder a <<Watson>>, un sistema de inteligencia artificial que ofrece múltiples posibilidades. Una de ellas es la de analizar los <<tonos>> en un determinado texto, resultando útil para controlar los mensajes que son enviados. Una vez analizado y devuelto el resultado del texto del mensaje, este debe ser interpretado por la aplicación para poder ser mostrado de una manera adecuada al administrador del chat.

\subsection{Objetivo IV: Integración de la aplicación con Google Calendar}
Este objetivo se centrará en el estudio de cómo compenetrar la aplicación con otros servicios como \textit{Google Calendar}, de manera que se puedan agregar nuevos eventos de calendario sin que el usuario tenga que cambiar de aplicación manualmente o realizar una inserción tediosa de información adicional. Ejemplos de estos eventos podrían ser añadir nuevos exámenes, reuniones, tutorías con los profesores, etc.

\subsection{Objetivo V: Implementar mecanismos que permitan comunicaciones personales profesor-tutores del alumno y viceversa}
Además de la comunicación mediante los grupos creados por el personal del centro, se podrán crear <<salas>> de chat integradas por un docente y una familia. Es decir, un docente crearía un grupo de chat para comunicarse únicamente con un núcleo familiar.

% Local Variables:
%  coding: utf-8
%  mode: latex
%  mode: flyspell
%  ispell-local-dictionary: "castellano8"
% End: