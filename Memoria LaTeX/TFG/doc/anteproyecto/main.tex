\documentclass{pre-tfg}

%\showhelp  % comenta o borra para eliminar ayudas

\title{Herramienta de mensajería instantánea en el ámbito educativo}
\author{Diego Andérica Richard}
\advisorFirst{Luis Rodríguez Benítez}
\advisorDepartment{Departamento de Tecnologías y Sistemas de Información}
\advisorSecond{Luis Jiménez Linares}
\intensification{TECNOLOGÍAS DE LA INFORMACIÓN}
\docdate{2018}{Marzo}

\begin{document}
	
\renewcommand{\tablename}{Tabla}

\maketitle
\tableofcontents

\newpage

%El anteproyecto recogería, en un \textcolor[rgb]{0.5,0.0,0.0}{máximo de 10 páginas},
%los siguientes apartados:
%
%\begin{itemize}
%\item Introducción (muy recomendable aunque no obligatorio)
%\item Tecnología específica cursada por el alumno
%\item Objetivos
%\item Método y fases de trabajo
%\item Medios que se pretenden utilizar
%\item Bibliografía básica consultada en la elaboración del anteproyecto
%\item Contrato de propiedad intelectual (si lo hubiera)
%\end{itemize}

\section{INTRODUCCIÓN}

%El capítulo de introducción podrá abordar los siguientes aspectos:
%
%\begin{itemize}
%\item Introducción al tema, entorno en el que el trabajo desempeñará
%  su objetivo, justificación de la importancia del trabajo abordado.
%\item Motivación y antecedentes (con algunas referencias bibliográficas).
%\item Descripción gráfica del proyecto (es aconsejable incorporar una figura que describa
%  el trabajo a desarrollar y que mejore la comprensión del mismo).
%\end{itemize}

En la actualidad se pueden encontrar numerosos avances en la tecnología que nos rodea simplificando, muchas veces, algunas tareas o funciones de la vida cotidiana. Un ejemplo de estos avances son los \textit{smartphones}, dispositivos con los que convivimos cada vez más y para más cosas en el día a día. Según un estudio publicado en 2017, el 81\% de los españoles usa un teléfono móvil inteligente, mientras que en 2012 la cifra era del 41\% \cite{Justo2017}, lo que representa un gran cambio en la sociedad y un considerable aumento en su adopción. Estos teléfonos posibilitan que el usuario esté permanentemente conectado a Internet pudiendo navegar, consultar información, realizar compras o realizar tareas tan sencillas como comunicarse con los demás. Es en este ámbito donde se pueden producir ciertos malentendidos que, a veces, simplemente pasan por no poner un carácter de más o porque la mensajería instantánea permite decir cosas que, en realidad, no queremos decir. Estos malentendidos podrían incluso llegar a afectar a los hijos que se encuentren en centros educativos puesto que, en este entorno, los padres suelen crear grupos de chat para hablar entre ellos, pudiendo surgir problemas como desprecios o que se comparta resuelta la tarea que el profesor ha mandado a los alumnos para que el resto, simplemente, la copie con la consecuente falta de aprendizaje. Normalmente se usan aplicaciones <<generalistas>> para este fin, aunque también se dispone de plataformas educativas como <<Papás 2.0>>, perteneciente a la comunidad autónoma de Castilla-La Mancha y que facilita la gestión administrativa a la vez que establece una vía de comunicación entre los centros educativos y las familias \cite{JCCM2017}.

No obstante, Papás 2.0 se encuentra más cercana al correo electrónico que a la mensajería instantánea, por lo que los usuarios suelen decantarse por la segunda opción puesto que está más extendida y, quizá, les resulte más fácil de utilizar. Algunos de los principales ejemplos de estas aplicaciones pueden ser \textit{WhatsApp} \cite{WhatsApp}, \textit{Telegram} \cite{Telegram2017} o \textit{Signal} \cite{Signal}. Las diferencias entre estas alternativas residen en algunas de las funciones que cada una ofrece, además del protocolo de cifrado para mantener los mensajes seguros de miradas <<ajenas>>.

Por tanto, se considera de especial relevancia el desarrollo de una aplicación de mensajería instantánea para \textit{smartphones} basados en el sistema operativo Android especialmente orientada al sector educativo, además de una plataforma web desde la que se pueda realizar una administración sencilla de usuarios y grupos por parte del centro.

\newpage

\section{TECNOLOGÍA ESPECÍFICA}
Este Trabajo Fin de Grado se ha desarrollado bajo la intensificación de \textit{Tecnologías de la Información}.

\begin{table}[hp]
	\centering
	\caption{Tecnología Específica cursada por el alumno}
	\label{tab:tec-especifica}
	
	\zebrarows{1}
	
	\begin{tabular}{p{0.6\textwidth}p{0.1\textwidth}}
		\textbf{Marcar la tecnología cursada} \\
		\hline
			Tecnologías de la Información & X\\
			Computación & \\
			Ingeniería del Software & \\
			Ingeniería de Computadores & \\
		\hline
	\end{tabular}

\end{table}

\begin{table}[hp]
	\centering
	\caption{Justificación de las competencias específicas abordadas en el TFG}
	\label{tab:competencias}
	\zebrarows{1}
	
	\begin{tabular}{p{0.3\linewidth}p{0.6\linewidth}}
		\textbf{Competencia} & \textbf{Justificación} \\
		\hline
		
			Capacidad para comprender el entorno de una organización y sus necesidades en el
			ámbito de las tecnologías de la información y las comunicaciones. & En este caso, esta competencia será útil para decidir las funcionalidades que la aplicación debe poseer tras interiorizar lo que resulta verdaderamente importante en una plataforma de este tipo para un sector tan concreto. \\
			
			Capacidad para seleccionar, desplegar, integrar y gestionar sistemas de información
			que satisfagan las necesidades de la organización, con los criterios de coste y calidad
			identificados. & Se deberá investigar sobre las tecnologías y servicios que ofrecen diferentes productos en cuanto a varios aspectos como bases de datos, almacenamiento o autenticación de usuarios y que su precio sea fácilmente abordable. \\
			
			Capacidad de concebir sistemas, aplicaciones y servicios basados en tecnologías de
			red, incluyendo Internet, web, comercio electrónico, multimedia, servicios interactivos
			y computación móvil. & Esta competencia será de utilidad puesto que, además del desarrollo de una aplicación para \textit{smartphones} con sistema operativo Android, se dispondrá de una plataforma web con la que se podrá interactuar para llevar la gestión y administración de la aplicación, por lo que el <<lado web>> será utilizado por la aplicación, siendo complementario.\\
			& \\
		\hline
	\end{tabular}

\end{table}

%El Trabajo Fin de Grado (TFG, de ahora en adelante) siempre deberá demostrar la aplicación
%de las competencias generales de la titulación. Además, el TFG deberá aplicar
%\textbf{algunas} de las competencias específicas asociadas a la \textbf{Tecnología
%  Específica o Intensificación} que el alumno ha cursado. Por lo tanto, el alumno incluirá
%en el anteproyecto \textbf{dos tablas}. Una tabla para seleccionar la tecnología cursada y
%en la que se contextualiza el TFG:
%
%\begin{table}[hp]
%  \centering
%  \caption{Tecnología Específica cursada por el alumno}
%  \label{tab:tec-especifica}
%
%  \zebrarows{1}
%  \begin{tabular}{p{0.6\textwidth}}
%    \textbf{Marcar la tecnología cursada} \\
%    \hline
%    Tecnologías de la Información \\
%    Computación \\
%    Ingeniería del Software \\
%    Ingeniería de Computadores \\
%    \hline
%  \end{tabular}
%\end{table}
%
%
%\clearpage
%
%En la segunda tabla, el alumno deberá justificar cómo \textbf{algunas}
%de las competencias específicas de la intensificación se aplicarán o
%tomarán forma en el TFG, \textbf{La relación de competencias por
%  intensificación se encuentran en el Anexo I al final de este
%  documento. }
%
%
%\begin{table}[hp]
%  \centering
%  \caption{Justificación de las competencias específicas abordadas en el TFG}
%  \label{tab:competencias}
%
%  \zebrarows{1}
%  \begin{tabular}{p{0.2\linewidth}p{0.7\linewidth}}
%    \textbf{Competencia} & \textbf{Justificación} \\
%    \hline
%    Competencia 1 & [Exponer y argumentar cómo y en qué parte se va a
%    abordar esta competencia en el TFG]\\
%    & \\
%    & \\
%    & \\
%    \hline
%  \end{tabular}
%\end{table}

\newpage

\section{OBJETIVOS}
Como se ha comentado anteriormente en la introducción, el uso de las aplicaciones de mensajería instantánea generalistas puede derivar en ciertos problemas entre los integrantes de los grupos, por lo que este trabajo fin de grado tiene como objetivo principal evitar estos problemas en un ámbito tan acotado como lo es el educativo. Por tanto, se considera de interés implementar una herramienta de mensajería instantánea para la comunicación entre el profesorado y los padres de los alumnos que trate de minimizar la ocurrencia de situaciones no deseadas, así como un sistema de gestión sencillo vía Web desde el que se pueda administrar dicha aplicación. Toda esta plataforma se valdrá de los servicios suministrados por Google con su producto \textit{Firebase}, que proporciona un \textit{backend} sencillo y fácil de utilizar, así como el entorno de desarrollo \textit{Android Studio}, puesto que la aplicación está destinada a móviles con sistema operativo \textit{Android}. Asimismo, este objetivo principal tiene algunos objetivos específicos.

\subsection{Objetivo I: Implementar un marco de gestión de usuarios vinculado al contexto educativo}
Se pretende implementar un mecanismo de gestión de usuarios que sea lo más eficiente y sencillo de utilizar. Por ejemplo, existe la posibilidad de que el personal del centro realice importación de archivos .csv para la creación de los grupos de chat. De igual manera, se deberán elegir y fijar diferentes roles, que serán asignados a las personas que utilicen la aplicación: el rol de moderador que estará destinado, principalmente, a los tutores, profesores o personal del centro que use la aplicación y que será el encargado de validar finalmente el envío de mensajes y el rol de usuario normal, destinado a los padres que se encuentren registrados.

\subsection{Objetivo II: Proporcionar un entorno de ejecución multiplataforma}
Puesto que el personal del centro tendrá las funciones de importar datos, registrar a los usuarios, crear y mantener los grupos de chat, éstas se realizarán mediante el uso de un ordenador personal puesto que, de esta manera, dichas tareas se tornan más sencillas de realizar y de un \textit{frontend} que resulte amigable y fácil de utilizar por parte del profesorado y personal del centro. Por otra parte, el resto de los usuarios de la aplicación accederán a la misma mediante el teléfono móvil o correo electrónico, puesto que los usuarios podrán elegir el método de entrada.

\subsection{Objetivo III: Implementar un mecanismo de monitorización activa del tipo y contenido de los mensajes}
Se busca disponer de algún tipo de monitorización para detectar mensajes que puedan ser potencialmente improcedentes dentro del contexto educativo. Esto se puede conseguir mediante el uso de la plataforma BlueMix de IBM, desde que la que se podrá acceder a «Watson», un sistema de inteligencia artificial con el que se pueden controlar los mensajes que son enviados a través de la aplicación. Una vez enviado el texto a analizar, éste debe ser interpretado por la aplicación, puesto que es devuelto en un formato diferente. Además, se dispondrá de cierta moderación puesto que el profesor responsable de cada grupo tendrá la capacidad de validar los mensajes que el resto de usuarios envíen a dicho grupo.

\subsection{Objetivo IV: Integración de la aplicación con Google Calendar}
Este objetivo se centrará en el estudio de cómo compenetrar la aplicación con otros servicios, como Google Calendar, de manera que se puedan agregar nuevos eventos de calendario sin que el usuario tenga que cambiar de aplicación manualmente. Ejemplos de estos eventos podrían ser añadir nuevos exámenes, reuniones, tutorías con los profesores, etc.

\subsection{Objetivo V: Implementar mecanismos que permitan comunicaciones personales profesor-tutores del alumno y viceversa}
Junto con la comunicación mediante los grupos creados por el personal del centro, los usuarios de la aplicación podrán iniciar una conversación con los profesores mediante la creación de un chat privado. Del mismo modo, los profesores podrán crear chats privados con el resto de usuarios.

%De acuerdo a la Introducción, el alumno deberá especificar cuál o cuáles son las hipótesis
%de trabajo de las que se parten, qué se pretende resolver, y en base a eso formular el
%objetivo principal del TFG.
%
%El objetivo principal deberá desglosarse en sub-objetivos parciales. Los sub-objetivos
%deberán describirse de forma breve y concisa.
%
%Como preámbulo a la formulación del objetivo parcial, el alumno deberá discutir sobre las
%limitaciones y condicionantes a tener en cuenta en el desarrollo del TFG (lenguaje de
%desarrollo, equipos, madurez de la tecnología, etc.).
%
%Del mismo modo, será recomendable incluir una lista preliminar de requisitos del sistema a
%construir.

\clearpage

\section{MÉTODO Y FASES DE TRABAJO}
%Para el desarrollo del proyecto, el alumno deberá seguir algún proceso o metodología afín
%al problema que pretende resolver. Para ello, deberá aportar una pequeña descripción del
%proceso o metodología (no más de una página) y \textbf{justificar su adecuación al
%  problema a resolver}.
%
%Del mismo modo, el alumno podrá realizar una breve planificación de la ejecución del
%proyecto según el proceso o metodología seleccionada.
%
%Como parte de la descripción del método y las fases de trabajo, el alumno podrá incluir
%una descripción preliminar de las tareas, una planificación temporal, diagramas de Gantt o
%recursos similares que pueda considerar necesarios.
%
%Si hubiera más de una metodología que a juicio del alumno podría ser afín al proyecto,
%éstas deberán mencionarse, y justificar la que considera más adecuada (esto puede
%considerarse parte de la justificación a la adecuación al problema a resolver).

En primer lugar, como metodología de gestión de proyectos, se ha decidido utilizar Scrum, que se trata de un marco de trabajo de procesos usado para la gestión de desarrollo de productos dentro del que se pueden emplear diferentes procesos y técnicas. De igual manera, se compone de equipos autogestionados con sus respectivos roles, eventos, artefactos y reglas \cite{Schwaber2017}. Siguiendo los principios y características de esta metodología, el proyecto deberá dividirse en diferentes fases, de tal manera que una fase no puede dar comienzo mientras la anterior no haya finalizado. Algunos de los términos más importantes en esta metodología son el Sprint, que es un bloque de tiempo donde se crea un entregable del producto y la Pila de Producto (\textit{Product Backlog}), que se compone de todo lo que podría ser necesario o relevante en el producto. Scrum, al basarse en la teoría de control de procesos empírica, asegura que el conocimiento procede de la experiencia de la toma de decisiones basada en lo que ya se conoce, empleando un enfoque iterativo e incremental. Sus tres pilares fundamentales son la transparencia, la inspección y la adaptación. De esta manera, se asegura que los aspectos significativos del proceso son visibles para los responsables del resultado, que los usuarios deben inspeccionar con frecuencia los artefactos y el progreso para detectar variaciones indeseadas y que, si un inspector determina que uno o más aspectos del proceso se desvían, se procederá a un reajuste tan pronto como sea posible, siendo comunicado al resto del equipo de Scrum.

Cada uno de los equipos autoorganizados y multifuncionales de Scrum se compone del \textbf{Dueño del Producto} (\textit{Product Owner}), el \textbf{Equipo de Desarrollo} (\textit{Development Team}) y un \textbf{<<Maestro de Scrum>>} (\textit{Scrum Master}). El Dueño del Producto se encarga de maximizar el valor del producto en cuanto al negocio, actúa de intermediario entre el cliente y el equipo y controla la Pila de Producto fijando sus ítems, ordenándolos y asegurándose de que cada uno se encuentra correctamente descrito. El Equipo de Desarrollo está formado por profesionales que entregan un incremento del producto terminado, es decir, la suma de los elementos de la lista de producto completados durante un Sprint. Por último, el Maestro de Scrum es el responsable de que la metodología se entienda y se adopte y de que el equipo sea productivo, siendo un <<facilitador>>. 

Por último, en cuanto a la metodología de desarrollo de software, se ha elegido la metodología <<iterativo e incremental>>, que consiste en desarrollar por partes el prodructo, integrándolas progresivamente conforme se van completando, agregando más funcionalidad al sistema final.

\section{MEDIOS QUE SE PRETENDEN UTILIZAR}

\subsection{Medios Hardware}
Principalmente, se va a usar un ordenador que ejecuta el sistema operativo Windows para llevar a cabo la consecución del proyecto y cuyas características técnicas más destacadas son:

\begin{itemize}
	\item \textbf{Marca y modelo:} Sony VAIO F-Series.
	\item \textbf{Procesador:} Intel\textregistered{ } Core\texttrademark{ } i7-720QM @ 1,6 GHz.
	\item \textbf{RAM:} 8 GB.
	\item \textbf{Tarjeta Gráfica:} NVIDIA GeForce GT 330m.
	\item \textbf{Disco Duro:} 500 GB.
\end{itemize}

Por otra parte, para realizar comprobaciones sobre un dispositivo real de la aplicación Android, se ha usado un \textit{smartphone} con las siguientes características:

\begin{itemize}
	\item \textbf{Marca y modelo:} LG Optimus L5 II.
	\item \textbf{Procesador:} MediaTek MT6575 @ 1 GHz.
	\item \textbf{Sistema Operativo:} Android 4.1.2 \textit{Jelly Bean}.
	\item \textbf{RAM:} 1 GB.
	\item \textbf{Memoria interna:} 4 GB.
\end{itemize}

\subsection{Medios Software}

\subsubsection*{Sistemas Operativos}
Como sistemas operativos, se han usado Microsoft Windows 10 Home \cite{Microsoft} en el PC y en cuanto al \textit{smartphone}, se usará Android en su versión 4.1.2 \textit{Jelly Bean} \cite{Andro}.

\subsubsection*{Lenguaje de Programación}
Puesto que la aplicación está destinada a \textit{smartphones} Android, el lenguaje de programación ha de ser Java. Java es un lenguaje de programación orientado a objetos usado para el desarrollo de aplicaciones cuyos propósitos son muy variados, puesto que también ofrece concurrencia \cite{Java}, así como HTML y JavaScript para el desarrollo de la parte Web de gestión usando, en este caso, el entorno de programación \textit{Eclipse Oxygen} \cite{EclipseFoundation2018}.

\clearpage

\subsubsection*{GitHub}
GitHub es una plataforma que ofrece la posibilidad de crear repositorios para proyectos y así poder trabajar de manera sencilla en colaboración con otras personas, como podrían ser los diferentes integrantes del equipo de Scrum. También dispone de un apartado para cada repositorio llamada \textit{projects} en la que se pueden crear tableros Kanban, que serán útiles durante el desarrollo del trabajo. Kanban \cite{Gomez2017} es una palabra de origen japonés que significa signo, señal o tarjeta. Este tablero resulta de gran ayuda puesto que se pueden observar de un rápido vistazo las tareas que quedan por hacer, en las que se está trabajando y las terminadas de una manera visual, organizada y rápida.

\subsubsection*{Android Studio}
Android Studio \cite{AndroidStudio} es el entorno de programación oficial para el desarrollo de aplicaciones en Android, proporcionando las herramientas necesarias para ello. Además, posee integración con \textit{Firebase}, lo que permite conectar las aplicaciones con este servicio para agregar \textit{Analytics}, \textit{Authentication} y \textit{Cloud Firestore }, entre otros servicios, que resultarán imprescindibles para el desarrollo de este trabajo.

\subsubsection*{Firebase}
Firebase \cite{GooFirebase} es, principalmente, un \textit{backend} que facilita las tareas de programación en el lado del servidor, puesto que proporciona acceso fácil a los recursos que ofrece. No sólo ofrece soporte al desarrollo de aplicaciones en Android, sino que también está disponible para su integración en iOS y aplicaciones Web. Algunas de sus funciones son:

\begin{itemize}
	\item \textbf{\textit{Cloud Firestore}}. Se trata de una base de datos en tiempo real, evolución de \textit{Realtime Database}. Ofrece una base de datos no relacional.
	\item \textbf{\textit{Authentication}}. Permite autenticar usuarios de forma simple en las aplicaciones de un proyecto. Además del usual método de entrada usando correo y contraseña, permite la autenticación mediante redes sociales y/o número de teléfono.
	\item \textbf{\textit{Remote Config}}. Permite modificar la aplicación de manera remota en todos los clientes sin necesidad de implementar una nueva versión.
\end{itemize}

\clearpage

Este servicio posee tres maneras de tarificación:

\begin{itemize}
	\item \textbf{Plan Spark}. Este es el plan más sencillo, con un coste gratuito. Posee ciertas limitaciones, aunque sería suficiente para un centro con un número de usuarios pequeño/medio ya que dichas limitaciones no son muy restrictivas.
	\item \textbf{Plan Flame}. Este plan tiene un coste de 25\$, con lo que se eliminan las restricciones del plan básico y sería apropiado para un centro con un número de usuarios relativamente grande.
	\item \textbf{Plan Blaze}. El último plan no tiene un precio definido puesto que se paga por lo que se vaya a utilizar, por lo que es el más flexible de los tres mencionados.
\end{itemize}

\subsubsection*{IBM Bluemix}
IBM Bluemix es una plataforma que permite el acceso a sus utilidades \textit{cloud} de manera sencilla. Ofrece diversos servicios entre los que se encuentra IBM Watson, que ofrece tecnologías cognitivas para crear aplicaciones inteligentes aportando la posibilidad de analizar y comprender sentimientos o palabras claves a partir de un texto. A la hora de interpretar si un mensaje es adecuado o no para su envío se utilizará el módulo \textit{Tone Analyzer} \cite{IBM}.

\subsubsection*{LaTeX}
En cuanto a la documentación, se ha usado el lenguaje de generación de documentos \LaTeX{}, junto con la clase proporcionada \textit{esi-tfg} \cite{ARCO}. \LaTeX{} es un sistema de preparación de documentos de alta calidad tipográfica usado principalmente en documentos técnicos o científicos y permite a los autores centrarse más en el contenido \cite{TheLatexProject}.

\clearpage

\section{REFERENCIAS}
%En esta sección se incluirán todas las referencias bibliográficas, ordenadas
%alfabéticamente por el primer apellido del primer autor, de las obras de las cuales se
%haya realizado alguna cita en los apartados anteriores. Las referencias deberán contener
%datos básicos como nombre y apellidos de los autores, título de la obra, evento al que
%pertenece, páginas, fecha y lugar de celebración (si se tratara de artículos de congreso),
%ISBN, editorial y ciudad (si se tratara de libro), nombre de revista, páginas, volumen y
%número (si se tratara de revista), etc.
%
%Se empleará un formato de referencia reconocido en el ámbito académico como
%ACM\footnote{http://www.acm.org/sigs/publications/proceedings-templates}\footnote{http://www.cs.ucy.ac.cy/\~{}chryssis/specs/ACM-refguide.pdf}.
%Otros formatos aconsejables son, por ejemplo, IEEE, AMA, APA y AMA.
%
%Las referencias son MUY importantes en el anteproyecto. Debes \textbf{citar} en la
%introducción al menos 6 o 7 artículos o revistas de investigación que traten el tema de
%principal y otros aspectos esenciales del proyecto.
%
%A continuación una sección de «Referencias» con ejemplos de referencias con formato ACM para:
%
%\begin{itemize}
%	\item Un artículo de revista~\cite{Bow93}.
%	\item Un informe técnico~\cite{Ding97}.
%	\item Un libro~\cite{Tavel07}.
%	\item Un capítulo de libro~\cite{Greiner99}.
%	\item Un artículo en las actas de un congreso~\cite{Frohlic00}.
%	\item Para una página web~\cite{Steele04} (con autores conocidos).
%	\item Para una página web~\cite{Oxygen} (con autores desconocidos).
%\end{itemize}

\bibliographystyle{alpha}
\singlespacing
\bibliography{main}

%\section{CONTRATO DE PROPIEDAD INTELECTUAL (si lo hubiera)}

%\newpage
%\section*{ANEXO I: Descripción de Competencias por Intensificación o Tecnología
%Específica\footnote{Este anexo se deberá borrar y no deberá ser incluido en el documento de anteproyecto final}}
%
%\subsection*{Intensificación de Computación}
%
%\begin{itemize}
%\item Capacidad para tener un conocimiento profundo de los principios fundamentales y
%  modelos de la computación y saberlos aplicar para interpretar, seleccionar, valorar,
%  modelar, y crear nuevos conceptos, teorías, usos y desarrollos tecnológicos relacionados
%  con la informática.
%\item Capacidad para conocer los fundamentos teóricos de los lenguajes de programación y
%  las técnicas de procesamiento léxico, sintáctico y semántico asociadas, y saber
%  aplicarlas para la creación, diseño y procesamiento de lenguajes.
%\item Capacidad para evaluar la complejidad computacional de un problema, conocer
%  estrategias algorítmicas que puedan conducir a su resolución y recomendar, desarrollar e
%  implementar aquella que garantice el mejor rendimiento de acuerdo con los requisitos
%  establecidos.
%\item Capacidad para conocer los fundamentos, paradigmas y técnicas propias de los
%  sistemas inteligentes y analizar, diseñar y construir sistemas, servicios y aplicaciones
%  informáticas que utilicen dichas técnicas en cualquier ámbito de aplicación.
%\item Capacidad para adquirir, obtener, formalizar y representar el conocimiento humano en
%  una forma computable para la resolución de problemas mediante un sistema informático en
%  cualquier ámbito de aplicación, particularmente los relacionados con aspectos de
%  computación, percepción y actuación en ambientes entornos inteligentes.
%\item Capacidad para desarrollar y evaluar sistemas interactivos y de presentación de
%  información compleja y su aplicación a la resolución de problemas de diseño de
%  interacción persona computadora.
%\item Capacidad para conocer y desarrollar técnicas de aprendizaje computacional y diseñar
%  e implementar aplicaciones y sistemas que las utilicen, incluyendo las dedicadas a
%  extracción automática de información y conocimiento a partir de grandes volúmenes de
%  datos.
%\end{itemize}
%
%
%\subsection*{Intensificación de Ingeniería de Computadores}
%
%\begin{itemize}
%\item Capacidad de diseñar y construir sistemas digitales, incluyendo computadores,
%  sistemas basados en microprocesador y sistemas de comunicaciones.
%\item Capacidad de desarrollar procesadores específicos y sistemas empotrados, así como
%  desarrollar y optimizar el software de dichos sistemas.
%\item Capacidad de analizar y evaluar arquitecturas de computadores, incluyendo
%  plataformas paralelas y distribuidas, así como desarrollar y optimizar software para las
%  mismas.
%\item Capacidad de diseñar e implementar software de sistema y de comunicaciones.
%\item Capacidad de analizar, evaluar y seleccionar las plataformas hardware y software más
%  adecuadas para el soporte de aplicaciones empotradas y de tiempo real.
%\item Capacidad para comprender, aplicar y gestionar la garantía y seguridad de los sistemas informáticos.
%\item Capacidad para analizar, evaluar, seleccionar y configurar plataformas hardware para
%  el desarrollo y ejecución de aplicaciones y servicios informáticos.
%\item Capacidad para diseñar, desplegar, administrar y gestionar redes de computadores.
%\end{itemize}
%
%
%\subsection*{Intensificación de Ingeniería del Software}
%
%\begin{itemize}
%\item Capacidad para desarrollar, mantener y evaluar servicios y sistemas software que
%  satisfagan todos los requisitos del usuario y se comporten de forma fiable y eficiente,
%  sean asequibles de desarrollar y mantener y cumplan normas de calidad, aplicando las
%  teorías, principios, métodos y prácticas de la Ingeniería del Software.
%\item Capacidad para valorar las necesidades del cliente y especificar los requisitos
%  software para satisfacer estas necesidades, reconciliando objetivos en conflicto
%  mediante la búsqueda de compromisos aceptables dentro de las limitaciones derivadas del
%  coste, del tiempo, de la existencia de sistemas ya desarrollados y de las propias
%  organizaciones.
%\item Capacidad de dar solución a problemas de integración en función de las estrategias,
%  estándares y tecnologías disponibles.
%\item Capacidad de identificar y analizar problemas y diseñar, desarrollar, implementar,
%  verificar y documentar soluciones software sobre la base de un conocimiento adecuado de
%  las teorías, modelos y técnicas actuales.
%\item Capacidad de identificar, evaluar y gestionar los riesgos potenciales asociados que pudieran presentarse.
%\item Capacidad para diseñar soluciones apropiadas en uno o más dominios de aplicación
%  utilizando métodos de la ingeniería del software que integren aspectos éticos, sociales,
%  legales y económicos.
%\end{itemize}
%
%
%\subsection*{Intensificación de Tecnologías de la Información}
%
%\begin{itemize}
%\item Capacidad para comprender el entorno de una organización y sus necesidades en el
%  ámbito de las tecnologías de la información y las comunicaciones.
%\item Capacidad para seleccionar, diseñar, desplegar, integrar, evaluar, construir,
%  gestionar, explotar y mantener las tecnologías de hardware, software y redes, dentro de
%  los parámetros de coste y calidad adecuados.
%\item Capacidad para emplear metodologías centradas en el usuario y la organización para
%  el desarrollo, evaluación y gestión de aplicaciones y sistemas basados en tecnologías de
%  la información que aseguren la accesibilidad, ergonomía y usabilidad de los sistemas.
%\item Capacidad para seleccionar, diseñar, desplegar, integrar y gestionar redes e
%  infraestructuras de comunicaciones en una organización.
%\item Capacidad para seleccionar, desplegar, integrar y gestionar sistemas de información
%  que satisfagan las necesidades de la organización, con los criterios de coste y calidad
%  identificados.
%\item Capacidad de concebir sistemas, aplicaciones y servicios basados en tecnologías de
%  red, incluyendo Internet, web, comercio electrónico, multimedia, servicios interactivos
%  y computación móvil.
%\item Capacidad para comprender, aplicar y gestionar la garantía y seguridad de los sistemas informáticos.
%\end{itemize}

\end{document}


% Local Variables:
% coding: utf-8
% mode: flyspell
% ispell-local-dictionary: "castellano8"
% mode: latex
% End:
